% Plantilla realizada por Alberto Brunete (UPM). Basada en la de Santiago Morante Cendrero (UC3M)

%parametros de tipo libro
\documentclass[10pt,a4paper]{book}

%idioma espa�ol y acentos
\usepackage[spanish, USenglish, es-tabla]{babel}
\usepackage[latin1]{inputenc}

%algunos s�mbolos matem�ticos y paquete para usar subim�genes
\usepackage{amsmath}
\usepackage{amsfonts}
\usepackage{amssymb}
\usepackage{graphicx}
\usepackage{subfigure}
\usepackage{listings}
\usepackage{appendix}
\usepackage{cite}
\usepackage{url}
%M�rgenes
\usepackage[left=3cm,top=3cm,right=3cm,bottom=3cm]{geometry}

%
\usepackage{multicol}
\usepackage{enumerate} % enumerados

%para generar �ndice con hiperv�nculos
\usepackage{hyperref}

%parametros del documento (sus propiedades)
\hypersetup{
    pdftitle={Nombre del alumno - TFG - a�o},
    pdfsubject={TFG - a�o},
    pdfauthor={Nombre del alumno},
    pdfkeywords={palabraclave1} {palabraclave2} {palabraclave3},
    colorlinks,
    citecolor=black,
    filecolor=black,
    linkcolor=black,
    urlcolor=black,
}

\usepackage[backend=biber,style=alphabetic]{biblatex}
\addbibresource{bibliografia/bibliografia.bib}

\DeclareLabelalphaTemplate{
  \labelelement{
    \field[uppercase,final]{shorthand}
    \field[uppercase]{label}
    \field[uppercase,strwidth=3,strside=left,ifnames=1]{labelname}
    \field[uppercase,strwidth=1,strside=left]{labelname}
  }
  \labelelement{
    \field[strwidth=2,strside=right]{year}
  }
}

%empieza el documento
\begin{document}  

%elementos antes del trabajo en s� se meten dentro de frontmatter
\frontmatter

%cada incluye referencia a un archivo de tipo .tex
\begin{titlepage}
\begin{center}

%forma de introducir im�genes. el \\[0.5 cm] de final de l�nea introduce un salto de ese tama�o.
%width=1\textwidth indica el tama�o de la im�gen (valores entre 0-1). 
 \includegraphics[width=1\textwidth]{figuras/cabecera.png}  \\[0.5 cm]

\LARGE UNIVERSIDAD POLIT�CNICA DE MADRID \\ [1 cm]

\LARGE ESCUELA T�CNICA SUPERIOR DE INGENIER�A Y DISE�O INDUSTRIAL \\ [1 cm]

\LARGE Grado en Ingenier�a Electr�nica Industrial y Autom�tica\\ [1 cm]

\LARGE \textbf{TRABAJO FIN DE GRADO}\\[1 cm]

\Huge \textsc{Guiado de un robot m�vil basado en ROS y kinect}\\[1 cm]

\LARGE Autor: Daniel Manzaneque Amo \\[2 cm]

%flushleft alinea a la izquierda el texto

\begin{multicols}{2} 
\begin{flushleft} \Large
\emph{Cotutor:} Miguel Hernando\\
\emph{Departamento:} Electr�nica, Autom�tica e Inform�tica Industrial
\end{flushleft}

\begin{flushleft} \Large
\emph{Tutor:} Alberto Brunete\\
\emph{Departamento:} Electr�nica, Autom�tica e Inform�tica Industrial
\end{flushleft}

\end{multicols} 

%rellena de blanco el resto de la p�gina para escribir abajo del todo
\vfill

% Bottom of the page
{\large Madrid, Febrero 2016}

\end{center}
\end{titlepage}

\include{capitulos/firmas}

%Licencia opcional
%\include{capitulos/licencia}

\include{capitulos/evaluacion}
%espaciado entre parrafos 8mm mayor del habitual
\setlength{\parskip}{4mm}

\chapter{Agradecimientos}

Cuando una etapa en la vida termina normalmente uno se para y reflexiona sobre las metas y objetivos alcanzados y muchas veces descuida todo aquel esfuerzo y los momentos dif�ciles que se sortearon hasta poder cumplirlos. Quiero que estos agradecimientos sirvan para reconocer el apoyo de todas aquellas personas que han contribuido en mi vida llegado este punto y que me han ayudado de una u otra manera a superar todos los retos.\\

En primer lugar quiero agradecer el apoyo incondicional de mi familia durante todos estos a�os, en especial a mis padres por haberme dado la oportunidad de estudiar algo que me entusiasma para que en un futuro pueda dedicarme a ello. Gracias a ellos y a mi hermana por el apoyo en cada uno de los momentos complicados y por disfrutar conmigo de los logros y buenos momentos.\\

Quiero agradecer el apoyo de todos mis amigos pero especialmente a Samu, Manu y Sas por sus �nimos, los buenos ratos que hemos pasado y porque s� que apuestan por m� m�s de lo que hago yo a veces.\\

No quiero dejar pasar la oportunidad de agradecer a Andrea todo el esfuerzo y la paciencia que ha tenido conmigo desde los primeros a�os de universidad, su inestimable apoyo en los momentos de baj�n y todas las experiencias que he vivido junto a ella y que me han servido para creer como persona, afrontar nuevos retos y creer m�s en m� cada d�a. Creo que podemos decir que los dos hemos hecho un gran doble grado despu�s de todo.\\

Durante mis a�os de universidad he tenido la oportunidad de compartir espacio y tiempo con grandes compa�eros. Daniela, Isma, Luis y Gema, hab�is sido grandes compa�eros y sois unos grandes amigos. Gracias por los buenos momentos en clase y sobre todo fuera de ella, ojal� nuestras inquietudes nos lleven a trabajar juntos en futuros proyectos.\\

Tambi�n deseo dar las gracias a los que han sido mis compa�eros de laboratorio durante el proyecto, Mario, Irene, Sof�a, Koldo y Chechu, por todas las tardes que hemos pasado juntos y por lo mucho que nos hemos re�do. S� que me acordar� de esos momentos m�s de una vez.\\

Agradecer a mis profesores de instituto, en especial a mis profesores de tecnolog�a, por hacer que su labor de ese�anza me llevase a estudiar este grado. Tambi�n a mis tutores de proyecto Miguel y Alberto por haberme dado la oportunidad de desarrollar este proyecto, por haber contribuido a mi ense�anza como ingeniero y por todos sus buenos consejos.\\

Por �ltimo, agradecer a todos los que me han aguantado con mi robotito andando y haciendo experimentos de aqu� para all�. �Me lo he pasado en grande!


\selectlanguage{spanish}

%chapter introduce un nuevo cap�tulo
\chapter{Resumen}

Muchos robots aut�nomos surgen como herramienta para acceder a lugares donde el ser humano no puede o no conviene que acceda porque se encontrar�a en riesgo.

Un robot aut�nomo es por tanto una pieza fundamental en tareas de rescate, salvamento, inspecci�n, exploraci�n de entornos peligrosos o inaccesibles, como la exploraci�n en la superficie de otros planetas. Adem�s, tareas sociales como la asistencia a humanos en entornos p�blicos, la interacci�n con el entorno o una navegaci�n m�s segura, como es el caso de los coches aut�nomos, cada vez est�n tomando m�s relevancia en nuestro d�a a d�a.

Este proyecto de fin de grado trata sobre el guiado y control de un robot m�vil de cuatro ruedas, con un sistema motriz en configuraci�n skid-steer, equipado con una serie de sensores que permiten su orientaci�n y posicionado en el entorno as� como un sensor capaz de captar este en tres dimensiones y un sensor adicional que lo hace tan solo en dos dimensiones.

Los datos de los sensores sirven tanto para construir mapas en dos dimensiones del entorno del robot como para navegar por �l evitando obst�culos de manera din�mica. El robot es capaz de generar mapas de celdas en los que situar tanto los objetos est�ticos como los m�viles, calcular una trayectoria adecuada y dirigirse hasta un punto indicado evitando obst�culos interpuestos en su camino.

Todo esta informaci�n, procesado de datos, c�lculo de trayectorias y ejecuci�n de movimientos se realiza en un ordenador de abordo integrado en el propio robot utilizando el software Robot Operating System (conocido en rob�tica por sus siglas ROS), que nos ofrece una interfaz com�n para interconectar nuestro robot con los sensores y con los algoritmos de navegaci�n.

A parte de la navegaci�n aut�noma, tambi�n se ha incluido un sistema de telecontrol del robot mediante otro ordenador  externo y de un algoritmo de detecci�n frontal de objetos en 3 dimensiones (nubes de puntos) que puedan servirle como gu�a. De esta forma, el robot es capaz de navegar siguiendo el movimiento de una persona o de un robot que le preceda.

El robot Pioneer 3 AT es el robot m�vil que se ha empleado en este proyecto (Figura \ref{fig:esquema_robot1}) y sobre el que se ha trabajado de manera espec�fica para realizar las pruebas reales de este proyecto. A este robot se le incorporan un sensor l�ser de dos dimensiones (sensor Sick LMS100) y un sensor de tres dimensiones (sensor Kinect). El c�mputo de la navegaci�n se realiza en un ordenador compacto incorporado en el robot (Intel NUC NUC5i7RYH).


\begin{figure}[htp]
\centering
\includegraphics[width=0.8\textwidth]{figuras/esquema_robot.jpg}
\caption{Esquema del sistema rob�tico utilizado en el proyecto} \label{fig:esquema_robot1}
\end{figure}

Las consignas de navegaci�n se realizan mediante un ordenador externo cualquiera conectado a una red inal�mbrica o mediante consignas de voz, en las que se indica al robot las tareas de navegaci�n a realizar (avanzar, girar, seguir a una persona...) o un punto del entorno al que dirigirse.

Todas estas implementaciones est�n desarrolladas bajo el entorno ROS, lo cual permite a�adir funcionalidades de manera m�s r�pida y menos laboriosa, como es el caso del control mediante comandos de voz o la interacci�n mediante sonidos. Es el caso tambi�n del simulador de rob�tica Gazebo, que se integra como funcionalidad en ROS y que ha servido para testar el sistema y aportar las pruebas te�ricas pertinentes para luego aplicarlas en el robot real.

Para concluir, podemos decir que este proyecto se encarga de integrar ROS como sistema en un ordenador de abordo incorporado en el robot que permita conectarse con los sensores y realizar la construcci�n de mapas y navegaci�n aut�noma mediante el c�culo de mapas y trayectorias globales y locales, realizar los movimientos del robot, as� como reconocer consignas de voz o de teleoperaci�n.

\paragraph{Palabras clave:} robot m�vil, ROS, navegaci�n reactiva, c�lculo de trayectorias.

\chapter{Abstract}

Achieving navigation and guidance of mobile robot comes up as a tool for rescue purposes in places where humans can't access or that involve a high risk for life. Many of those repetitive and fatigating tasks could be done with a robust and capable mobile robot.

An autonomous robot is, an essential part in rescue, inspection and exploration tasks developed in dangerous or non-reacheable places, such as the surface of other planets. Moreover, social tasks such as assitance for humans in pubic places, interaction with the environment or a safer navigation in the cities. Autonomous cars are a good example of this.

This final degree project is about guidance and control of a four-wheel mobile robot with a skid-steer configuration. It is equipped with a sort of sensors, allowing it to make positioning and orientation in the environment. There is also a main sensor capturing the enviaronment in three dimensions and an additional one doing it in two dimensions.

Sensor data is used to build two dimensional maps of the exploration place as well as to take care of dynamical obstacles. The robot can build maps formed by cells where to incorpore or raytrace static and dynamic obstacles, calculate the proper trajectory plan and head for a destination point avoiding obstacles in its way.

All this information, data processing, trajectory calculation and movement execution is done in an onboard computer inside the robot. It uses the Robot Operating System software (known as ROS), which offers a commmon interface to communicate the robot with sensors and navigation algorithms.

Apart from autonomous navigation, the robot also has a telecontrol system from an outside computer and an algorithm to detect frontal objects in three dimensions (pointclouds) that can guide the robot. This is how it can navigate following a person when it is walking or another robot in front of it.

Pioneer 3 AT robot is the one used in this project (Figure \ref{fig:esquema_robot2}). It has been the specific platform for all real tests. This robot is equipped with a two dimension laser scanner (Sick LMS100 sensor) and a three dimensional sensor (Kinect sensor). The navigation computation is done in an onboard compact computer (Intel NUC NUC5i7RYH computer).

\begin{figure}[htp]
\centering
\includegraphics[width=0.8\textwidth]{figuras/esquema_robot.jpg}
\caption{Diagram of the robotic system used for this project} \label{fig:esquema_robot2}
\end{figure}

The navigation commands are sent from an outside computer connected to the same wireless network or from voice navigation commands speaking directly to the robot (go forward, backward, turn right...) or pointing a goal in the map.

All those implementations are developed under ROS framework. This is why additional features can be added in a faster and effortless way. Tha is the case of the robot simlator Gazebo, which integrates as an add-on in ROS. Gazebo has been used to perform tests in navigation and to check theorical concepts to lately incorporate them in the real robotic system.

To conclude, it can be said that this project integrates ROS as a robotic system in an onboard computer and connects to sensors to perform tasks such as building maps or navigation from one point to another. The system calculates local and global maps and trajectories, makes movements according to them, as well as recognises voice or teleoperation commands.

\paragraph{Keywords:} mobile robot, ROS, reactive navigation, trajectory calculation.

%genera �ndice
\tableofcontents

%�ndice de figuras.
\listoffigures

%�ndice de tablas.
\listoftables

%empieza la parte descriptiva del trabajo
\mainmatter

\chapter{Introducci�n}

Este proyecto surge como una manera de actualizar y poner en funcionamiento el robot m�vil de investigaci�n Pioneer 3 AT, integrarlo en una nueva plataforma rob�tica como es ROS e implementar la navegaci�n aut�noma del robot basada en sensores de percepci�n del entorno como el sensor Kinect y un sensor l�ser tipo LIDAR.

En este proyecto se desarrolla e implementa un sistema rob�tico tomando como base al robot Pioneer y sobre la plataforma ROS desde su inicio de forma que todos los sensores queden integrados dentro del robot y este se encuentre operativo para resolver tareas de navegaci�n. Todo ello con el prop�sito de servir como plataforma de navegaci�n en futuros proyectos relacionados con la rob�tica m�vil o con el entorno ROS en concreto.

\section{Motivaci�n del proyecto}
Dotar a un robot de la capacidad de navegar aut�nomamente puede ser una alternativa imprescindible en el caso de que se necesite explorar un entorno que no sea f�cilmente accesible para el ser humano o que conlleve cierto riesgo.

Cualquier proyecto que desarrolle la automatizaci�n de un proceso es ya una motivaci�n, puesto que se va a dise�ar una m�quina que sea capaz de realizar una tarea que antes solo pod�a realizarse por un ser humano. Adem�s, dichas tareas realizadas por un robot pueden realizase, en principio, con una mayor precisi�n y con mayor repetibilidad debido a que se elimina el factor del cansancio.

Este proyecto viene motivado por la integraci�n de ROS dentro de una plataforma m�vil. Con esta plataforma de desarrollo software podemos explorar un concepto diferente de programaci�n en rob�tica, que ofrece caracter�sticas como:
\begin{itemize}
    \item Abstraerse de la programaci�n a bajo nivel.
    \item Reutilizar software ya desarrollado.
    \item Interfaz de comunicaci�n com�n.
    \item Escalabilidad del sistema.
    \item Simulaci�n mediante Gazebo.
    \item Visualizaci�n gr�fica de la informaci�n aportada por sensores.
    \item Transformaci�n entre los diferentes sistemas de coordenadas.
\end{itemize}

Estas caracter�sticas hacen que adoptar ROS como plataforma de desarrollo rob�tico sea una ventaja en cuanto a la facilidad para integrar diferentes dispositivos de un robot, comunicarnos con �l a trav�s de TCP/IP, utilizar desarrollos ya existentes... Al ser ROS una plataforma muy viva y din�mica, con muchas personas utiliz�ndolo en universidades y empresas de muchos lugares del mundo, permite que el sistema evolucione e incorpore nuevas caracter�sticas que pueden ser f�cilmente integrables en futuras actualizaciones del robot.

Utilizar el sensor Kinect es otra de las motivaciones de este proyecto debido al bajo coste del mismo frente al gran aporte de informaci�n que supone percibir el entorno en tres dimensiones. Se puede realizar una navegaci�n basada solamente en este sensor adem�s de reconocer objetos por su forma y realizar el guiado del robot esquivando estos.

La aplicaci�n que m�s ha servido como motivaci�n para el desarrollo de este proyecto ha sido la automatizaci�n de las tareas de conducci�n de autom�viles \cite{cocheautonomo}, un sector que se encuentra en auge y que comienza a dar sus primeros pasos en el mundo real \cite{nevadagoogle}.

Tambi�n los robots de exploraci�n espacial de la NASA, en especial a su �ltimo rover en Marte, Curiosity \cite{marsNASA}, que permite explorar el entorno �rido de la superficie marciana con un alto grado de autonom�a en las labores de inspecci�n y an�lisis de elementos.

\section{Objetivos del proyecto}

El objetivo de este proyecto es realizar el control de un robot m�vil para que sea un robot aut�nomo, bas�ndose en la informaci�n que da la odometr�a, en la nube de puntos que proporciona un sensor que captura el entorno en 3 dimensiones como el sensor Kinect y en el sensor l�ser LIDAR. Este control debe realizarse con la ayuda del software ROS, integr�ndolo como parte del sistema rob�tico para que sirva de soporte al desarrollo del proyecto.

El robot debe ser capaz de localizarse y situarse en el entorno, el sensor Kinect el sensor l�ser aportar�n informaci�n sobre los objetos alrededor del robot, y el software desarrollado para ROS deber� ser capaz de hacer un control reactivo sobre los movimientos para permitir al robot moverse por interiores y guiarlo hacia un punto indicado.

A la finalizaci�n del proyecto, el sistema rob�tico debe quedar integrado de tal forma que en futuros desarrollos pueda utilizarse como plataforma m�vil y permita abstraerse de las tareas de orientaci�n y navegaci�n.\\

El alcance del proyecto requiere m�ltiples fases de trabajo:

En primer lugar, se necesita un conocimiento previo del sistema hardware, como es el robot Pioneer 3 AT as� como el sensor Kinect y el sensor l�ser. C�mo integrar estos elementos y acceder a la informaci�n que aportan sus sensores y comandar al robot para que realice movimientos.

En segundo lugar, requiere un conocimiento del entorno de desarrollo ROS. Las herramientas software de las que dispone, el funcionamiento interno y la manera de programar e interaccionar con los diferentes elementos, el aprendizaje y comprensi�n.

En tercer lugar, incorporar los sensores pertinentes para obtener la informaci�n que permita al robot posicionarse en el entorno.

En cuarto lugar, implementar los ajustes necesarios para que el robot pueda operar utilizando el entorno de navegaci�n ofrecido por ROS. Realizar una configuraci�n �ptima de los sensores y realizar las pruebas reales para el c�lculo de trayectorias y el control reactivo del robot.

Por �ltimo, realizar la integraci�n del sistema dentro de la plataforma rob�tica. Disponer de todo lo necesario para que el robot quede totalmente adaptado al sistema ROS e integrado con el sensor Kinect y el sensor l�ser.

\subsubsection{Objetivos espec�ficos}

A continuaci�n se citan los puntos m�s importantes para cumplir con el objetivo de este proyecto:
\newcounter{enum2} % creamos un contador par la enumeraci�n. 
\begin{enumerate}[a)]
\item Familiarizarse con el control de los robots m�viles y m�s concretamente en el control del robot Pioneer 3 AT. Familiarizarse con el sensor Kinect y con el software necesario para su uso. Y familiarizarse con el framework ROS y sus herramientas para el desarrollo de sistemas rob�ticos.
\setcounter{enum2}{\value{enumi}}
\end{enumerate}

\begin{enumerate}[a)]
\setcounter{enumi}{\value{enum2}}
\item Detecci�n de obst�culos simples con el sensor Kinect para su inclusi�n posterior en el sistema de navegaci�n del robot.
\setcounter{enum2}{\value{enumi}}
\end{enumerate}

\begin{enumerate}[a)]
\setcounter{enumi}{\value{enum2}}
\item Telecontrol del robot Pioneer 3 AT a partir del software ROS para realizar un controlador manual desde un ordenador externo.
\setcounter{enum2}{\value{enumi}}
\end{enumerate}

\begin{enumerate}[a)]
\setcounter{enumi}{\value{enum2}}
\item Realizar un sistema de navegaci�n aut�nomo que sea capaz de dirigir el robot bas�ndose en la informaci�n aportada por el sensor Kinect y otro tipo de sensores embebidos.
\setcounter{enum2}{\value{enumi}}
\end{enumerate}

De los puntos anteriores podemos desgranar algunas fases intermedias como son:

\newcounter{enum3}
\begin{enumerate}[a)]
\item Comprender el funcionamiento de ROS y la integraci�n e interacci�n del software desarrollado bajo este entorno.
\setcounter{enum3}{\value{enumi}}
\end{enumerate}

\begin{enumerate}[a)]
\setcounter{enumi}{\value{enum3}}
\item Control del movimiento del robot Pioneer 3 AT a trav�s de ROS, as� como la obtenci�n de la informaci�n de la odometr�a, estado de la bater�a, encendido de motores...
\setcounter{enum3}{\value{enumi}}
\end{enumerate}

\begin{enumerate}[a)]
\setcounter{enumi}{\value{enum3}}
\item Puesta en marcha el sistema de navegaci�n para robots de ROS conocido como ''Navigation Stack'' y exploraci�n de las capacidades del sistema.
\setcounter{enum3}{\value{enumi}}
\end{enumerate}

\begin{enumerate}[a)]
\setcounter{enumi}{\value{enum3}}
\item Valorar el uso de sensores adicionales y buscar una disposici�n �ptima de los mismos para integrarlos en el sistema de navegaci�n y en la arquitectura hardware del propio robot.
\setcounter{enum3}{\value{enumi}}
\end{enumerate}

\pagebreak

\section{Estructura del documento}
A continuaci�n y para facilitar la lectura del documento se detalla el contenido que consta en cada uno de los cap�tulos.

\begin{itemize}
\item En el cap�tulo 1 se realiza una introducci�n del proyecto, un an�lisis de las motivaciones y objetivos del mismo.
\item En el cap�tulo 2 se hace un repaso del estado del arte de la rob�tica m�vil, los tipos de robots y principales sensores utilizados as� como las aplicaciones de la rob�tica m�vil.
\item En el cap�tulo 3 se expone el planteamiento del proyecto, los pasos seguidos junto con un an�lisis de tiempos y las tecnolog�as utilizadas.
\item En el cap�tulo 4 se habla de los conceptos relacionados con ROS, su filosof�a de funcionamientoy sus herramientas as� como la estructura del desarrollo software llevado a cabo en este proyecto.
\item En el cap�tulo 5 se abordan los conceptos relacionados con la navegaci�n, su enfoque te�rico y las particularidades de funcionamiento de la navegaci�n en ROS.
\item En el cap�tulo 6 se exponen los primeros nodos de control a bajo nivel, el nodo de teleoperaci�n y el de navegaci�n estimada.
\item En el cap�tulo 7 se expone la implementaci�n del sistema pasando por todas sus fases de desarrollo y las soluciones que se han llevado a cabo.
\item En el cap�tulo 8 se exponen las configuraciones y el uso de los simuladores rob�ticos utilizados para dar apoyo al desarrollo del proyecto. 
\item En el cap�tulo 9 se exponen las pruebas del sistema en navegaci�n real y simulada, se exponen los resultados y se discuten los mismos.
\item En el cap�tulo 10 se exponen las conclusiones y se plantean los futuros desarrollos a implementar en el robot m�vil.
\end{itemize}

Adicionalmente existen 3 anexos que sirven de apoyo a las explicaciones del proyecto y donde se realiza un manual aclaratorio del funcionamiento del sistema rob�tico objeto de este proyecto.
  
  %partes finales del trabajo: conclusiones, bibliografia y anexos
  
\chapter{Estado del arte}

La rob�tica m�vil vive actualmente un momento de gran desarrollo para multitud de aplicaciones en entornos diversos, desde espacios abiertos con orograf�a accidentada y condiciones clim�ticas adversas **proyecto robot catastrofes** **proyecto DARPA big dog**, entornos controlados y espacios interiores conocidos como la automatizaci�n de tareas de almacenaje de productos **referencia robots amazon**, hasta orientaci�n y exploraci�n de espacios interiores desconocidos **robot creacion de mapas de edificios**.\\

De la misma forma, el inter�s en robots que sea capaces de reproducir las capacidades de un ser humano e incluso que pueda dar asistencia ya sea en entornos conocidos o no abre un �rea de posibilidades en las que los robots m�viles cobran importancia.\\

Los avances tanto el las caracter�sticas hardaware como software son notables aunque estas suelen variar dependiendo de la aplicaci�n a la que un robot est� destinado. En este cap�tulo trata de hacer un resumen del estadoactual de la rob�tica m�vil.\\


\section{Hardware en rob�tica m�vil}

Como hemos indicado previamente, la configuraci�n hardware de un robot m�vil var�a dependiendo de la aplicaci�n a la que vaya destinado. Es cierto que lo ideal para un robot ser�a disponer de una configuraci�n hardware com�n que fuera polivalente en los diferentes terrenos y situaciones, sin embargo, debido a la variedad de aplicaciones y dado que un robot suele destinarse a tareas espec�ficas, la elecci�n del hardware que mejor se adapta es una tendencia com�n en rob�tica.\\

Para seleccionar el hardaware debemos valorar el tipo de actuador que se requiere, entendi�ndose por actuador al dispositivo que genera el movimiento de los elementos que hacen que el robot m�vil se desplace. En rob�tica m�vil suelen utilizarse los actuadores de tipo ele�ctrico, ya que ofrecen unas prestaciones de potencia, controlabilidad y coste adecuados. Adem�s, ofrecen la posibilidad de que la alimentaci�n est� integrada en el robot, haci�ndolo independiente de una fuente de energ�a accesoria.\\

Los actuadores el�ctricos son, por tanto, los m�s utilizados. En concreto, los motores de corriente continua (Figura \ref{fig:motor_encoder}) ofrecen un f�cil control y f�cil acoplamiento a un encoder. Los encoder son sensores de posici�n que perminten conocer el giro de un eje de rotaci�n. Estos sensores son muy importantes en rob�tica m�vil, ya que a partir de la informaci�n que arrojan el robot tiene consciencia de su posici�n relativa, en el caso de los encoders incrementales, o su posici�n absoluta, en el caso de los encoders absolutos. **referencia a la secci�n de sesnsores**.\\

\begin{figure}[htp]
\centering
\includegraphics[width=0.4\textwidth]{figuras/motor_encoder.jpg}
\caption{Motor de corriente continua con encoder} \label{fig:motor_encoder}
\end{figure}

Existen otros tipos de actuadores el�ctricos que se utilizan en rob�tica, como puede ser el caso de los motores paso a paso, sin embargo su baja velocidad de giro no los hacen adecuados para robots m�viles.\\

La disposici�n de los actuadores determina la configuraci�n del robot. Centr�ndonos en robots que se desplazan mediante ruedas y descartando a los robots con patas, podemos distinguir las siguientes configuraciones: Ackerman, triciclo cl�sico, tracci�n diferencial, skid-steer, s�ncrona y omnidireccional.\\

**IMAGEN de las configuraciones LIBRO**

\newcounter{enum1} % creamos un contador par la enumeraci�n. 
\begin{enumerate}[a.]
  \item Configuraci�n Ackerman
\setcounter{enum1}{\value{enumi}} % le damos al contador el valor de la enumeraci�n.
\end{enumerate}

Consta de cuatro ruedas. Las ruedas motrices son las traseras o las delanteras, y �stas �ltimas se encargan adem�s de la direcci�n. Permite un desplazamiento a altas velocidades y la posibilidad de realizar giros con estabilidad. Esta configuraci�n es la que se utiliza en la industria del autom�vil.\\

\begin{enumerate}[a.]
\setcounter{enumi}{\value{enum1}} % reiniciamos enumeraci�n con el valor del contador.
  \item Triciclo cl�sico
\setcounter{enum1}{\value{enumi}}
\end{enumerate}

Consta de tres ruedas. Las ruedas motrices pueden ser las dos ruedas traseras o solo la delantera, que se encarga de la direcci�n. Este es el caso de los triciclos y de algunas bicicletas. Esta configuraci�n ofrece alto grado de maniobrabilidad penalizando la estabilidad del conjunto y realizar giros den 90�.\\

\begin{enumerate}[a.]
\setcounter{enumi}{\value{enum1}}
  \item Configuraci�n diferencial
\setcounter{enum1}{\value{enumi}}
\end{enumerate}

Consta de dos ruedas colocadas en el eje perpendicular a la direcci�n de desplazamiento del robot. Cada rueda es controlada por un motor, de tal forma que la diferencia de velocidad giro de una rueda respecto a otra determina el giro, avance o retroceso del robot. Los robots que presentan esta configuraci�n suelen utilizar una tercera rueda que gira libremente que sirver como apoyo (rueda loca). Es la configuraci�n t�pica de las sillas de ruedas y su caracter�stica principal es que permite realizar giros completos sobre s� mismo.\\

\begin{figure}[htp]
\centering
\includegraphics[width=0.3\textwidth]{figuras/pioneer3dx.jpg}
\caption{Robot de configuraci�n diferencial Pioneer 3 DX} \label{fig:pioneer_3dx}
\end{figure}

\begin{enumerate}[a.]
\setcounter{enumi}{\value{enum1}}
  \item Skid steer
\setcounter{enum1}{\value{enumi}}
\end{enumerate}

Consta de cuatro ruedas, todas ellas motrices, y su principio de funcionamiento es el mismo que el utilizado en la configuraci�n diferencial. Esta configuraci�n presenta las ventajas de la configuraci�n diferencial, pudiendo realizar giros sobre el eje del robot, pero presenta la desventaja de que las ruedas deben deslizarse lateralmente, por tanto existe un rozamiento que var�a en funci�n de la inclinaci�n el tipo de terreno que dificulta realizar un modelo cinem�tico.\\

Proporciona mucha tracci�n y estabilidad y suele encontrarse en aplicaciones relacionadas con la exploraci�n, veh�culos obra o veh�culos todo terreno (Figura \ref{fig:ejemplos_skid-steer}).\\

\begin{figure}[htp]
\centering
\includegraphics[width=0.7\textwidth]{figuras/skid-steer_examples.png}
\caption{Ejemplos de configuraci�n diferencial: Cargador frontal, Robotnik Guardian, Pioneer 3 AT} \label{fig:ejemplos_skid-steer}
\end{figure}

Este sistema es el que se utiliza tambi�n en los tanques de guerra, aunque en vez de neum�ticos se utilizan orugas, denominado configuraci�n por deslizamiento de cintas **Referencia**.\\

\begin{enumerate}[a.]
\setcounter{enumi}{\value{enum1}}
  \item Configuraci�n s�ncrona
\setcounter{enum1}{\value{enumi}}
\end{enumerate}

Conformado por tres o m�s ruedas acopladas mec�nicamente y dotadas de tracci�n, este sistema permite que todas las ruedas roten en la misma direcci�n y giren a la misma velocidad (Figura \ref{fig:configuracion_sincrona}). Es utilizada ampliamente en rob�tica para robots m�viles de interior, aunque est� siendo desplazada por la configuraci�n omnidireccional. \\

\begin{figure}[htp]
\centering
\includegraphics[width=0.4\textwidth]{figuras/configuracion_sincrona.png}
\caption{Configuraci�n s�ncrona} \label{fig:configuracion_sincrona}
\end{figure}

\begin{enumerate}[a.]
\setcounter{enumi}{\value{enum1}}
  \item Configuraci�n omnidireccional
\setcounter{enum1}{\value{enumi}}
\end{enumerate}

Consta de 3 ruedas cada una con un motor independiente, que permiten el desplazamiento en cualquier direci�n (Figura \ref{fig:configuracion_omnidireccional}). Las ruedas omnidireccionaes constan de una serie de rodillos con el eje de rotaci�n perpendicular a la direccio�n de avance.\\

\begin{figure}[htp]
\centering
\includegraphics[width=0.4\textwidth]{figuras/configuracion_omnidireccional.png}\hspace{2cm}
\includegraphics[width=0.3\textwidth]{figuras/rueda_omnidireccional.png}
\caption{Configuraci�n s�ncrona y rueda omnidireccional} \label{fig:configuracion_omnidireccional}
\end{figure}

Esta configuraci�n diferencial empieza a utilizarse en sistemas de 4 ruedas con las demonminadas ''Mecanum Wheels'' **REFERENCIAAA**, que son ruedas similares a las omnidireccionales pero con los rodillos colocados en cierto �ngulo (Figura \ref{fig:ruedas_mecanum}). La combinaci�n de los giros de cada una permiten al robot moverse en cualquier direcci�n.\\

\begin{figure}[htp]
\centering
\includegraphics[width=0.4\textwidth]{figuras/mecanum-wheels_uranus.png}
\caption{Robot Uranus con ruedas tipo Mecanum} \label{fig:ruedas_mecanum}
\end{figure}

\section{Sensores en rob�tica m�vil}

Para que un robot pueda realizar tareas con una determinada precisi�n y velocidad debe conocer el entorno del sistema en el que se quiera actuar as� como el estado del robot en ese sistema.\\

Existen dos tipos de sensores, los sensores internos, que aportan informaci�n sobre la posici�n orientaci�n del robot, y los externos, que aportan informaci�n del entorno alrededor del robot.

\subsection{Sensores internos}

Dentro de los sensores internos, los sensores de posci�n primordiales son los encoders, tanto los de tipo incremental como los de tipo absoluto (Figura \ref{fig:esquema_encoder}). Su funcionamiento se basa en un foto-emisor y un foto-receptor que detectanel paso o no de luz a trav�s de un disco con ciertas marcas acoplado al eje de giro del actuador.

\begin{figure}[htp]
\centering
\includegraphics[width=0.4\textwidth]{figuras/encoder_absoluto.png}
\caption{Esquema de un encoder absoluto} \label{fig:esquema_encoder}
\end{figure}

Los sensores de velocidad son similares a los encoders pero miden la velocidad de giro del eje del actuador. La tacogeneratriz proporciona una tensi�n proporcional a la velocidd de giro.\\

Los sensores aceler�metros o inclin�metros, permiten conocer la inclinaci�n del robot en cada uno de sus ejes, as� como las aceleraciones producidas por su propio desplazamiento.\\

Existen otros sensores m�s sofisticados como las Unidades de medida inercial (IMU) (Figura \ref{fig:sensor_imu}). Son disposiivos que combinan las medidas de un gir�scopo y varios aceler�metros para determinar la posici�n relativa (x, y, z) y la orientaci�n (roll, pitch, yaw), velocidad y aceleraci�n respecto a un sistema de referencia.\\

\begin{figure}[htp]
\centering
\includegraphics[width=0.4\textwidth]{figuras/sensor_imu.png}
\caption{Unidad de medida inercial, IMU} \label{fig:sensor_imu}
\end{figure}

Debido  a  que  la  aceleraci�n  se  ha  de  integrar  dos  veces  para  obtener  la 
posici�n,  el  error  crece  de  forma  cuadr�tica.  Luego  para  largos  periodos  de operaci�n las unidades IMU se deben de resetear con otros sensores tipo GPS.\\

\subsection{Sensores externos}

Los sensores externos son aquellos que nos aportan informaci�n sobre el estado del robot respecto al entorno o que nos da informaci�n sobre lo que ocurre alrededor de este.\\

Los sensores de presencia, como son los sensores de tipo inductivo, capacitivo, �ptico o mec�nico. Sea cual sea la naturaleza del sensor, su funci�n es la de detectar presencia. Un ejemplo de aplicaci�n a un robot m�vil ser�a una serie de sensores de presencia mec�nicos, denominados "fin de carrera", colocados en la parte delantera, de modo que al tocar alg�n obst�culo se tuviera conciencia de la presencia de un obst�culo.\\

Sensores de posicionamiento global GPS (Global Positioning System) que permiten determinar la posici�n de un objeto en todo el mundo, normalmente con una precisi�n de metros. El GPS funciona con una red de sat�lites con trayectorias sincronizadas que cobren toda la superficie de la tierra. El GPS lanza se�ales a los sat�lites y calculando el tiempoq ue tarda �stos en responder, se obtiene la posici�n por trianguaci�n.\\

Los sensores GPS se utilizan en robots m�viles que operan en el exterior y suen combinarse con otros sensores que ofrezcan una mayor precisi�n.\\

Los sensores de distancia son aquellos que nos dan una referencia de la longitud que existe a los objetos cercanos. Es el caso de los sensores e ultrasonidos, donde un emisor emite una onda ultras�nica y cuando es reflejada por un objeto se puede determinar la distancia a la que se encuentra midiendo el tiempo que tarda el sonido en ir y volver. Los sensores de distancia tambi�n pueden ser infrarrojos, funcionando de la misma manera.\\

Existen sensores de distancia que utilizan tecnolog�a l�ser para determinar la longitud de un punto a otro, se denominan Scanners l�ser. Estos sensores emiten rayos l�ser en un plano de 2 dimensiones y en un rango determinado, y midiendo el tiempo de vuelo del haz l�ser son capaces de obtener una medida muy precisa de la distancia.\\

Existen otro tipo de sensores de distancia que permiten obtener distancias a puntos de manera tridimensional. Algunos utilizan un sistema de doble c�mara conocidos como c�mara estereosc�pica (Figura \ref{fig:camara_estereoscopica}). Estos sesnores son capaces de obtener im�genes 3D con la informaci�n de dos im�genes tomadas a cierta distancia una de otra. Es el sensor m�s parecido a la visi�n humana.\\

\begin{figure}[htp]
\centering
\includegraphics[width=0.4\textwidth]{figuras/pr2_stereocam.jpg}
\caption{C�mara estereosc�pica del robot PR2} \label{fig:camara_estereoscopica}
\end{figure}

Otros sensores de distancia en 3 dimensiones, son los sensores de tipo Infrarrojo **BUSCAR NOMBRE CORRECTO SENSORES 3D BOSCH**, como es el sensor Kinect, que ha se ha vuelto muy popular debido a su bajo coste y su buena respuesta.\\

Kinect es un dispositivo derrollado por PrimeSense**refenrecia** y distribuido por Microsoft para la vieoconsola Xbox 360 (Figura \ref{fig:sensor_kinect}).\\

Inicialmente permit�a controlar e interactuar con la consola XBOX sin necesidad de tener contacto f�sico con un controlador. Este sensor permite reconocer gestos, comandos de voz, objetos e im�genes; esto hace que tenga mucho inter�s en el mundo de la rob�tica.

\begin{figure}[htp]
\centering
\includegraphics[width=0.4\textwidth]{figuras/kinect.jpg}
\caption{Sensor de 3 dimensiones Kinect para Xbox 360} \label{fig:sensor_kinect}
\end{figure}

Para captar el entorno en 3 dimensiones, Kinect incluye una c�mara de v�deo RGB, un emisor de haz infrarrojo y una c�mara infrarroja.

\section{Control en la rob�tica m�vil actual}

El control del movimento en los robots m�viles con ruedas puede describirse, de manera general, en cuatro tareas fundamentales: localizaci�n y orientaci�n, planificaci�n de trayectoria, seguimiento de la misma y evasi�n de los obst�culos.\\

\subsection{Localizaci�n y orientaci�n en un entorno}
Normalmente, uno de los mayores problemas que conciernen a la navegaci�n de robots m�viles consiste en la determinaci�n de su localizaci�n respecto a un mapa en funci�n de la informaci�n captada por los sensores.. No basta con situar una referencia global, si no que es imprescindible conocer la posici�n relativa respecto a los posibles obst�culos, tanto m�viles como est�ticos, de su entorno. Para esta tarea existen diferentes opciiones, como utilizar mapas introducidos en el robot, o bien elaborar un mapa de manera simult�nea al movimiento del robot por un lugar, como si se tratase de un robot de exploraci�n. Es lo que se conocde como SLAM (Simultaneous Localization and Mapping)**refenrecia**.\\

Uno de las motivaciones para el uso de esta t�cnica es la construcci�n de mapas desde el punto de vista del robot, as� como el ruido que se genera en los sensores de posici�n internos del robot que miden la odometr�a. Sin embargo, esta t�cnica en ocasiones puede producir efectos no deseados, como incorrecciones en el mapa debido a su alto coste computacional o variaciones debidas a objetos que se mueven en torno al robot.\\

Pueden distenguirse tres tipos de mapas: geom�etrico, topol�gico y sem�ntico. El nivel geom�trico es el m�s utilizado y consiste en un mapa m�trico donde se representan los segmentos b�asicos de un entorno, o un mapa discretizado, donde se efectua la descomposici�n de los elementos en celdillas. En el nivel topol�gico, se representaran nodos y coneciones entre ellos, y el nivel sem�antico es cuando se elimina la informaci�n geom�trica.

\section{Aplicaciones actuales de la rob�tica m�vil}

Algunas de las aplicaciones del �rea de la rob�tica m�vil ya han sido mencionadas con anterioridad en este documento, estas van enfocadas a sustituir la labor que realiza el ser humano en situaciones de riesgo o en tareas repetitivas que aportan poco valor.\\

Una de las aplicaciones m�s famosas sobre rob�tica m�vil son los Rovers de exploraci�n espacial de la NASA ''Spirit'' y ''Opportunity'' dentro de la misi�n ''Mars Exploration Rover'' lanzada en 2003. Estos robots disponen de sistemas de navegaci�n y exploraci�n ideados para sus misiones en la superficie del planeta Marte, con el objetivo de analizar el entorno y los materiales de sus rocas y cr�teres y enviar la informaci�n de vuelta a la Tierra. **REFERENCIA**\\

Un tercer robot no tripulado fue enviado con posterioridad a la superficie del planeta rojo. El robot ''Curiosity'' forma parte de la segunda generaci�n de robots de exploraci�n espacial y aterriz� en Marte en el a�o 2012. La misi�n ''Mars Science Laboratory'' ha permitido descubrir la existencia de antiguos lagos en la superficie del planeta **referencia pagina NASA**.\\

\begin{figure}[htp]
\centering
\includegraphics[width=0.6\textwidth]{figuras/curiosity_rover.jpg}
\caption{Imagen tomada a s� mismo por el robot Curiosity en la superficie marciana} \label{fig:curiosity_rover}
\end{figure}

Otras aplicaciones conocidas de la rob�tica m�vil son los robots de b�squeda, reconocimiento y desactivaci�n de explosivos. Robots como el iRobot 510 PackBot (Figura \ref{fig:robots_explosivos}), desarrollado por iRobot Corporation, o TALON (Figura \ref{fig:robots_explosivos}), desarrollado por QinetiQ North America, son ejemplos de c�mo la rob�tica m�vil es una herramienta muy valiosa en situaciones peligrosas o en sectores como seguridad y defensa.\\

\begin{figure}[htp]
\centering
\includegraphics[width=0.4\textwidth]{figuras/irobot510.jpg}
\includegraphics[width=0.4\textwidth]{figuras/talon.jpg}
\caption{Robots de desactivaci�n de explosivos: iRobot 510 Packbot y TALON} \label{fig:robots_explosivos}
\end{figure}

Tambi�n existen robots ideados para realizar tareas en entornos peligrosos o en situaciones de desastre. Como ejemplo podemos citar la cat�strofe que sufr�o Jap�n el 11 de Marzo de 2011. Un terremoto causa da�os catastr�ficos en la central nuclear de Fukushima **FUENTE DE LA NOTICIA**. Los niveles de radiaci�n son tan altos que solo los robots son capaces de entrar a valorar la situaci�n de la central.\\

Estos robots, preparados para aguantar la raciaci�n y realizar tareas de exploraci�n y limpieza son Quince (Figura \ref{fig:quince_raccoon_robots}), desarrolado por Chiba Institute of Technology **referencia**, un robot m�vil capaz de subir y bajar escaleras y operar en las plantas superiores de la central nuclear.

Y Raccoon (Figura \ref{fig:quince_raccoon_robots}), desarrollado por Tepco, equipado con dos cabezales m�viles preparados para aspirar y limpiar, encargado de recuperar el polvo contaminado del edificio del reactor 2.

\begin{figure}[htp]
\centering
\includegraphics[width=0.4\textwidth]{figuras/quince_robot.jpg}
\includegraphics[width=0.5\textwidth]{figuras/raccoon_robot.jpg}
\caption{Robot de exploraci�n de desastres Quince} \label{fig:quince_raccoon_robots}
\end{figure}

Siguiendo con la aplicaci�n de robots m�viles en tareas de limpieza, podemos destacar la popularizaci�n de los robots dom�sticos de tipo aspiradora, que se encargan de las tareas repetitivas del entorno del hogar, como el iRobot Roomba (Figura \ref{fig:roomba_robot}).

\begin{figure}[htp]
\centering
\includegraphics[width=0.5\textwidth]{figuras/roomba_robot.png}
\caption{Robot para limpieza del hogar Roomba, de iRobot} \label{fig:roomba_robot}
\end{figure}

Tambi�n existen aplicaciones de robots m�viles en el �bito de la agricultura. La empresa Robotnik**refenrecia** tiene en marcha diferentes programas para incorporar sus robots en tareas de recolecci�n o inspecci�n de la cosecha.\\

Su proyecto, AgriRobot investiga el aspecto de la interacci�n humano-robot (Human-Robot Interface, HRI, en ingl�s). Para ello se sirven del robot Summit X Lincorporado con 4 ruedas motoras de alta potencia. El robot contiene un pulverizador el�ctrico de 10 litros, tiene un sistema de visi�n, navegaci�n y localizaci�n, y utiliza el software ROS.

Al igual que el proyecto VinBot, de la misma empresa. Un robot m�vil aut�nomo todo terreno dotado con un conjunto de sensores capaces de capturar y analizar im�genes de vi�edos y datos en 3D mediante el uso de aplicaciones de cloud computing. Su finalidad es determinar el rendimiento de los vi�edos y compartir esta informaci�n con los viticultores.

Transporte urbano: coches aut�nomos.\\



\chapter{Alcance y objetivos del proyecto}

En este cap�tulo se define el alcance y los objetivos de este proyecto, es decir, lo que se pretende conseguir con este proyecto y hasta donde puede llegar.\\

\section{Prop�sito y alcance}

El prop�sito de este proyecto es el control autom�tico de un robot m�vil utilizando ROS. Lo que se pretende es implementar la navegaci�n aut�noma del robot bas�ndose en un control reactivo a partir de la informaci�n obtenida a trav�s del sensor Kinect.\\

El alcance del proyecto requiere m�ltiples elementos de trabajo:\\

En primer lugar, requiere un conocimiento previo del sistema hardware, como es el robot Pioneer 3 AT as� como la sensor Kinect. C�mo integrar estos elementos y acceder a la informaci�n que aportan sus sensores y comandar al robot para que realice movimientos.\\

En segundo lugar, requiere un conocimiento del entorno de desarrollo ROS. Las herramientas software de las que dispone, el funcionamiento interno y la manera de programar e interaccionar con los diferentes elementos, el aprendizaje y comprensi�n.\\

En tercer lugar, incorporar los sensores pertienentes para obtener la informaci�n que permita al robot posicionarse en el entorno.\\

En cuarto lugar, implementar los ajustes necesarios para que el robot pueda operar utilizando el entorno de navegaci�n ofrecido por ROS. Realizar una configuraci�n �ptima de los sensores y realizar las pruebas reales para el c�culo de trayectorias y el control reactivo del robot.\\

Por �ltimo, realizar la integraci�n del sistema dentro de la plataforma rob�tica. Disponer de todo lo necesario para que el robot quede totalmente adaptado al sistema ROS e integrado con el sensor Kinect y los sensores pertinentes.

\section{Objetivos}

El objetivo de este proyecto es realizar el control de un robot m�vil para que sea un robot aut�nomo, bas�ndose en la informaci�n que da la odometr�a y la nube de puntos que proporciona un sensor que captura el entorno en 3 dimensiones como el sensor Kinect.


\chapter{Desarrollo del proyecto}

En esta cap�tulo se expone cu�l ha sido el planteamiento del proyecto y los pasos que se han seguido para conseguir los objetivos y llegar a unos resultados �ptimos.

\section{Planteamiento}
El robots sobre el que se pretende trabajar es el Pioneer 3 AT, de la empresa Adept Mobile Robots, cuyas caracter�sticas se detallar�n m�s adelante. La configuraci�n del sistema motriz es de tipo skid-steer y ser� determinante a la hora de realizar el control del desplazamiento.

Para realizar la teleoperaci�n del robot, utilizaremos las herramientas de comunicaci�n de ROS, que hacen que la ejecuci�n de los diferentes nodos de forma distribuida entre equipos se realice de forma transparente para el usuario. Con esta caracter�tica podremos desarrollar con facilidad un sistema de telecontrol sin preocuparnos en exceso por la implementaci�n de la comunicaci�n entre equipos.

Para la navegaci�n se pretende que el robot base sus movimientos en un sistema reactivo, es decir, que el robot base su navegaci�n principalmente en la informaci�n captada por sus sensores y no en un mapa preestablecido. El control en navegaci�n del robot se basar� en la funcionalidad "Navigation Stack" de ROS, que tambi�n ser� explicada en detalle m�s delante.

El desarrollo principal para la navegaci�n se basa en la infomaci�n aportada por el sensor Kinect, sin embargo, el sensor l�ser proporciona una informaci�n muy potente para robots y tambi�n ha sido incluido en el dasarrollo de este proyecto, utiliz�ndolo en conjunto con el sensor Kinect.

Seguidamente, se han realizado los ajustes pertinentes en la navegaci�n del robot, para la cual se ha seguido el concepto de mapas de coste y descomposici�n en celdas. Tambi�n se ha valorado la disposici�n de ambos sensores para capturar el entorno, as� como el tratamiento dispar de los datos capturados por cada uno de ellos. De esta forma logramos que no se produzcan detecciones de objetos de manera duplicada y que no haya discrepancias entre los obst�culos que detecta un sensor respecto al otro \footnote{De especial inter�s en la icorporaci�n de obst�culos al mapa mediante el uso de Costmaps en el sistema de navegaci�n de ROS}.

Finalmente, se han incorporado caracter�sticas adicionales que aportan valor al desarrollo del proyecto, como el uso del simulador Gazebo o la interacci�n con el robot mediante comandos de voz y sintetizado de voz.

\section{Planificaci�n del proyecto}

En este apartado se desarrollan las fase por las que ha pasado ete proyecto y realizaremos un an�lisis de tiempos.

Fase inicial: Familiarizaci�n con el entrono ROS y elecci�n de herramientas.

\begin{enumerate}[i.]
  \item Utilizaremos las herramientas proporcionadas por ROS para evaluar los datos que pueda manejar el robot: RViz, rqt\_grapth, map\_server, rostopics...
  
  \item Para el control del movimiento del robot utilizaremos el nodo RoSAria debido a sus amplias posibilidades.
  
  \item Para acceder a la informaci�n de la Kinect utilizaremos los driver libfreenect, por ser librer�as de c�digo libre y utilizadas ampliamente.
\end{enumerate}

Segunda fase: Realizaci�n del nodo de teleoperaci�n y comunicaci�n entre equipos conectados a la misma red.

\begin{enumerate}[i.]
  \item Utilizamos la configuraci�n de equipos en red para acceder a la informaci�n publicada por nodos que se ejecuten en varias m�quinas **referencia**
  
  \item Partindo del nodo de teleoperacion de "Turtlesim" *refrencia*, realizamos un nodo similar para nuestro robot.
\end{enumerate}

Tercera fase: Incorporaci�n de los sensores al robot y acceso a los datos.

\begin{enumerate}[i.]
  \item Para el sensor Kinect, realizamos un adaptador para conectarlo a la alimetnaci�n del robot. Utilizando el nodo "freenect\_stack" *REFERENCIA*, accedemos a la nube de puntos y la imagen.
  
  \item Utilizamos el nodo LMS1xx **Referencia** para la puesta en marcha del l�ser y el acceso a los datos.
  
  \item Incorporaci�n de sistemas de referencia "base\_link", "laser", ''camera\_link'' y sus transformadas mediante el paquete "tf" **referencia**. 
  
  \item Visualizaci�n del conjunto de datos junto con los ejes de referencia en RViz.
\end{enumerate}

Cuarta fase: Incorporaci�n del sistema de navegaci�n y ajuste de los par�metros

\begin{enumerate}[i.]
  \item Calibrado de los encoders de las ruedas del robot y ajuste de la odometr�a mediante RosAria.
  
  \item Incorporaci�n del sistema de navegaci�n ROS de forma b�sica.
  
  \item Navegaci�n utilizando el sensor Kinect y el sensor Sick y generado de mapas mediante "slam\_gmapping".
  
  \item Ajuste de los planeadores de taryectoria del robot y par�metros de giro y control.
\end{enumerate}

Quinta fase: Ajuste de la navegaci�n y simulaci�n mediante gazebo

\begin{enumerate}[i.]
  \item Navegaci�n en modo global (utilizando un mapa guardado) y en modo local (completamente reactivo).
  
  \item Puesta en marcha del simulador Gazebo y configuraci�n del robot en el entorno.
  
  \item Disposici�n de los sensores de manera �ptima y remodelado de la estructura f�sica del robot.
\end{enumerate}

Sexta fase: Nuevas funcionalidades y toma de datos.

\begin{enumerate}[i.]
  \item Incorporaci�n de la funcionalidad "follower", adaptada a partir del robot Turtlebot **Refeencia**, para el guiado del robot.
  
  \item Interfaz de comandos por voz y sintetizador de texto a voz.
  
  \item Pruebas f�sicas, recogida y an�lisis de los datos.
\end{enumerate}


An�lisis de tiempos:

Este proyecto fin de grado comenz� en Noviembre de 2014 y termin� en Febrero de 2015.

Durante el primer mes de Noviembre se estuvo recopilando informaci�n sobre ROS y su funcionamiento, los desarrollos existentes aplicados a robots reales y la filosof�a del sistema.

En el mes de Diciembre se comenz� a trabajar con el robot, comprobando que todos los elementos se encontraban en correcto funcionamiento y se instal� el sistema operativo en su ordenador de abordo

Duarante el mes de Enero se pudo avanzar menos debido a los ex�menes y trabajos de las �ltimas asignaturas.

En el mes de Febrero se retom� el trabajo, empezando por una primera toma de contacto con la librer�a Aria y la ejecuci�n de movimientos desde un ordenador externo conectado v�a puerto serie.

Durante los meses de Marzo y Abril, el robot comenz� a funcionar con ROS, realizando los primeros movimientos con control por teclado. Seguidamente se realiz� el nodo de telecontrol y un nodo para realizar movimientos basados tan solo en la odometr�a.

En el mes de mayo, se comenzaron a probar la compatibilidad con ROS de la c�mara Kinect y el sensor L�ser. Acto seguido, comenzaron las primeras pruebas de navegaci�n aut�noma.

En los meses de Junio y Julio, siguieron los ajustes en la navegaci�n, tanto en el planificador de trayectoria como en los mapas de coste, as� como en el sensor Kinect para la detecci�n de obst�culos a diferente altura. Adem�s se incorpor� la funcionalidad de seguimiento.

En Julio tambi�n comenzaron las primeras pruebas de comandos de voz y la sintetizaci�n de voz.

Durante ese mes y el mes de Agosto, se comenz� a redactar gran parte del trabajo en esta memoria, donde se organiz� la estructura del proyecto y la informaci�n a incluir.

En el mes de Septiembre se decidi� incorporar un ordenador m�s potente al robot y reestructurar su chasis para dejar el sistema desarrollado integrado de manera permanente. Tambi�n se utiliz� el array de micr�fonos del sensor Kinect para los comandos de voz.

Duante el mes de Octubre se organiz� la estructura del proyecto y se puso en marcha el simulador Gazebo. A continuaci�n se realiz� el ajuste de los sensores y la optimizaci�n del sistema de navegaci�n. En paralelo se realizadon las modificaciones mec�nicas y estructurales para la integraci�n de los sensores y el ordenador en el robot.

Durante el mes de Novimebre se realiz� un peque�o par�n a nivel de software, se continu� con la parte mec�nica y con la redacci�n de la memoria de este proyecto.

A continuaci�n, comenzaron a realizarse las pruebas reales con la nueva configuraci�n en el robot.

A continiaci�n se muestra un diagrama de Gantt con el an�lisis de tiempos de las diferentes tareas.

\begin{landscape} 
\begin{ganttchart}{1}{48}
  \gantttitle{2014}{6}
  \gantttitle{2015}{36}
  \gantttitle{2016}{6} \\
  \gantttitlelist{11,...,12}{3}
  \gantttitlelist{1,...,12}{3}
  \gantttitlelist{1,2}{3} \\
  \ganttgroup{Entorno ROS}{1}{3} \\
  \ganttbar{Recabar informacion}{1}{2} \\
  \ganttlinkedbar{Realizar tutoriales}{2}{3} \\
  \ganttgroup{Pruebas robot}{4}{12} \ganttnewline
  \ganttbar{Sistema Operativo}{4}{4} \\
  \ganttlinkedbar{Librer�a Aria}{7}{8} \\
  \ganttlinkedbar{Primeros movimientos}{9}{12} \\
  \ganttgroup{Pruebas ROS}{13}{18} \ganttnewline
  \ganttbar{Instalacion ROS}{13}{14} \\
  \ganttlinkedbar{Telecontrol}{15}{16} \\
  \ganttlinkedbar{Movimientos odometr�a}{17}{19} \\
  \ganttgroup{Navegacion}{19}{40} \ganttnewline
  \ganttbar{Navegacion b�sica}{19}{20} \\
  \ganttlinkedbar{Navegacion}{21}{28}{29}{30} \\
  
  %\ganttlink{elem2}{elem3}
  %\ganttlink{elem3}{elem4}
\end{ganttchart}
\end{landscape}

\section{Tecnolog�as y herramientas empleadas en el proyecto}

En esta secci�n se describen las tanto las tecnolog�as como las herramientas utilizadas en el desarrollo del proyecto. 

\subsection{Robot Operating System}

El Sistema Operativo Rob�tico \cite{ROS} (conocido en ingl�s como Robot Operating System o ROS) es un framework para el desarrollo de software para robots que provee la funcionalidad de un sistema operativo \cite{quigley2009ros}. ROS fue desarrollado originalmente en 2007 por el Laboratorio de Inteligencia Artificial de Stanford para dar soporte al proyecto del Robot con Inteligencia Artificial de Stanford \cite{stair}. Desde 2008, el desarrollo continua primordialmente en Willow Garage, un instituto de investigaci�n rob�tico con m�s de veinte instituciones que colaboran conjuntamente.

ROS provee los servicios est�ndar de un sistema operativo como abstracci�n del hardware, control de dispositivos de bajo nivel, implementaci�n de funcionalidad de uso com�n, paso de mensajes entre procesos y mantenimiento de paquetes. Est� basado en una arquitectura de nodos interconectados que que pueden mandar, recibir y multiplexar mensajes de sensores, control, estados, planificaciones y actuadores, entre otros. La librer�a est� orientada para un sistema UNIX (Ubuntu (Linux)) aunque tambi�n se est� adaptando a otros sistemas operativos como Fedora, Mac OS X, Arch, Gentoo, OpenSUSE, Slackware, Debian o Microsoft Windows, considerados como 'experimentales'.

ROS consta de dos partes b�sicas: la parte del sistema operativo, ros, como se ha descrito anteriormente y ros-pkg, una suite de paquetes aportados por la contribuci�n de usuarios (organizados en conjuntos llamado en ingl�s "stacks") que implementan la funcionalidades tales como localizaci�n y mapeo simult�neo, planificaci�n, percepci�n, simulaci�n, etc.

ROS ofrece principlamente dos lenguajes de programaci�n para acceder a su API (Application Programming Interface) completa. Esos lesnguajes son C++ y Python \cite{rosapi}. 

ROS es software libre bajo t�rminos de licencia BSD. Esta licencia permite libertad para uso comercial e investigador. Las contribuciones de los paquetes en ros-pkg est�n bajo una gran variedad de licencias diferentes.

Actualemtente ROS es mantenido y desarrollado de manera Open Source por Open Source robotics Foundation \cite{osrf}, una organizaci�n independiente sin �nimo de lucro fundada por miembros de la comunidad rob�tica a nivel global.

\subsection{Lenguaje de programaci�n C++}

El lenguaje C++ es un lenguaje orientado a objetos, y como tal, tiene como objetivo la reducci�n del tiempo de desarrollo aumentando la eficacia del proceso de generaci�n de los programas.

Como consecuencia, los programas tienden a tener menos l�neas de c�digo y con m�s facilidad de introducir elementos nuevos escritos
por otras personas.

Al tratarse de un lenguaje compilado, presenta una buena eficiencia en tiempo de ejecuci�n frente a los lenguajes interpretados.

En sistemas operativos basados en Linux, el lenguaje C++ se compila bajo el compilador GCC (GNU Compiler Collection).

Dentro del desarrollo software en C++ para ROS (roscpp \cite{roscpp}), existe una amplia interfaz para acceder a las diferentes funcionalidades y comunicarse con nodos desarrollados tanto en C++ como en Python.

\subsection{Lenguaje de programaci�n Python}
Python es un lenguaje de programaci�n interpretado cuya principal caracter�stica es que utiiza una sintaxis que favorece el c�digo legible.

Se trata de un lenguaje de programaci�n multiparadigma, ya que soporta orientaci�n a objetos, programaci�n imperativa y, en menor medida, programaci�n funcional. Es un lenguaje interpretado, usa tipado din�mico y es multiplataforma.

Gracias a sus caracter�sticas, su uso es totalmente flexible y permite un tiempo de desarrollo menor principalmente por su tipado din�mico� y su sintaxis. Sin embargo, al tratarse de un lenguaje interpretado, el tiempo de ejecuci�n es m�s alto lo cual no lo hace adecuado para tareas que requieran altos niveles de eficiencia.

Desntro del desarrollo en Python para ROS (rospy \cite{rospy}), existe una interfaz completa para comunicarse con los nodos y otras funcionalidades de ROS desarrolladas en Python o C++.

\subsection{Controlador de versiones git y repositiorios github}

Git es un software de control de versiones libre. Es decir, git
gestiona los archivos y directorios y los cambios hechos en ellos a lo largo del tiempo. Esto te permite recuperar antiguas revisiones del proyecto o ver tu historial de cambios.

Git fue creado pensando en la eficiencia y la confiabilidad del mantenimiento e versiones cuando estas tienen un gran n�mero de archivos de c�digo fuente. Tiene la capacidad de poder trabajar varias personas con el mismo paquete siempre que no modifiquen el mismo archivo, en ese caso, ser�a posible ver las diferencias entre ambas versiones, y unirlas o crear unarama del proyecto principal si fuera necesario tener las dos versiones.

GitHub es un sistema de almacenamiento p�blico de c�digo fuente (de cualquier tipo) o un servicio de respositorios. Su principal caracter�stica es la de ofrecer una plataforma de interacci�n social \cite{dabbish2012social}en la que distintas personas pueden trabajar en conjunto. Esto permite que varios desarrolladores contribuyan a un proyecto y trabajen de manera coordinada.

Tanto para el desarrollo software de este proyecto como para la redacci�n de esta memoria se han utilizado estas herramientas, y el acceso a los respositorios se ecuentra en las siguientes direcciones:

\begin{itemize}
	\item Desarrollo software del proyecto \url{https://github.com/danimtb/pioneer3at_ETSIDI}
	\item Memoria del proyecto \url{https://github.com/danimtb/TFG_pioneer3at}
\end{itemize}

\subsection{Simulador de rob�tica Gazebo}

Gazebo \cite{gazebo} es un simulador de rob�tica en tres dimensiones que ofrece la simulaci�n de complejos entornos de diversas caracter�sticas, as� como robots de todo tipo, su interacci�n con el entorno y la representaci�n visual de datos obtenidos por diversos sensores como c�maras, l�seres, ultrasonidos...

Un buen simulador de rob�tica es esencial para cualquier tipo de desarrollo rob�tico, ya que podemos realizar las pruebas software o la viabilidad de un sistema antes de construirlo. Gazebo cuenta con un potente motor de f�sica simulada, interacci�n con objetos y din�mica de los mismos \cite{koenig2004design}.

Gazebo permite una integaci�n completa con ROS, gestiona modelos f�sicos de robots utilizando el formato URDF (Unified Robot Description Format) \cite{urdf} y a�ade caracter�sticas espec�ficas como el tipo de material, los momentos de inercia o el modelo de colisi�n. Adem�s, incorpora plug-ins (funcionalidades a�adidas) que permiten la simulaci�n de robots de tipo diferencial, simulaci�n de sensores y el c�lculo de tranformadas entre los distintos sistemas de referencia.

Gazebo es mantenido y desarrollado actualmente por la Open Source Robotics Foundation \cite{osrf}.

\subsection{RViz: Robot Visualization tool}


\section{Hardware}
En esta parte se explica detalladamente el hardware empleado en el desarrollo del proyecto.

\subsection{Pioneer 3 AT}

El robot Pioneer 3 AT (Figura \ref{fig:pioneer3at}), perteneciente a la empresa Adept MobileRobots, es un robot de cuatro ruedas en configuraci�n skid-steer y todo terreno (AT, All Terrain) de operaci�n e investigaci�n en laboratorio.\\

\begin{figure}[htp]
\centering
\includegraphics[width=0.4\textwidth]{figuras/pioneer_3_at.jpg}
\caption{Robot Pioneer 3-AT} \label{fig:pioneer3at}
\end{figure}

Su configuraci�n en skid-steer permite un control relativamente simple utilizando el modo diferencial para poder realizar giros con gran maniobrabilidad, sin embargo, esta configuraci�n depende mucho del tipo de suelo, con lo que se pierde precisi�n.

Este robot dispone de bater�as, interruptor con parada de emergencia, dos motores de corriente continua para cada par de ruedas con transmisi�n mediante correa, encoders para leer la odometr�a y un microcontrolador con firmware ARCOS.

Ademas cuenta con un peque�o computador interno conectado al microcontrolador que puede utilizarse para realizar operaciones de manera aut�noma.

El cuerpo del robot es de aluminio y su parte delantera as� como superior es f�cilmente desmontable para realizar las conexiones pertinentes y acceder al ordenador de a bordo y la placa microcontroladora. En la plataforma superior se sit�a el panel de control (Figura \ref{fig:panel_control})para acceder al ordenador de abordo conectando un monitor, teclado y rat�n, puerto serial RS-232, botones de encendido y reset varios leds indicadores de estado y de env�o y recepci�n de datos.

\begin{figure}[htp]
\centering
\includegraphics[width=0.6\textwidth]{figuras/panel_control.png}
\caption{Panel de control del robot Pioneer 3-AT} \label{fig:panel_control}
\end{figure}

En la siguiente tabla (Tabla \ref{tabla_pioneer3at}) se describen las principales caracter�sticas del robot.

% Please add the following required packages to your document preamble:
% \usepackage{graphicx}
\begin{table}[!h]
\centering
\resizebox{\textwidth}{!}{%
\begin{tabular}{c c}
\hline
{\bf Especificaciones} & {\bf Pioneer 3 AT} \\ \hline
Largo & 508 mm \\
Ancho & 497 mm \\
Alto & 277 mm \\
Distancia al suelo & 80 mm \\
Peso & 12 kg \\
Carga �til & 32 kg \\
Cuerpo & Aluminio de 1.6 mm \\
Bater�as & 3 de 12 V ~Ah, estancas, plomo-�cido \\
Autonom�a & 4-8 horas \\
Sistema motriz & 4 ruedas motrices \\
Ruedas & Neum�ticos de Nylon \\
Di�metro de rueda & 222 mm \\
Ancho de rueda & 88 mm \\
Sistema de giro & Diferencial \\
radio m�xima curvatura & 40 cm \\
Radio de giro & 0 cm \\
M�xima velocidad de avance & 1.2 m/s \\
M�ximo escal�n & 10 cm \\
M�ximo hueco & 15.2 cm \\
Terreno & Asfalto, Tierra, C�sped, etc. \\
Encoders & 500 pulsos \\
Procesador & Hitachi H8S \\ \hline
\end{tabular}
}
\caption{Especificaciones del robot Pioneer 3 AT}
\label{tabla_pioneer3at}
\end{table}


\subsection{Sensor Kinect}

Kinect es un conjunto de sensores de bajo coste que lo convierte en una
herramienta excepcional (Figura \ref{fig:sensor_kinect}). Este dispositivo incluye una c�mara de v�deo RGB, una c�mara infrarroja de profundidad, un array de micr�fonos y altavoces, un aceler�metro y un peque�o motor que le permite hacer movimientos de inclinaci�n.

\begin{figure}[htp]
\centering
\includegraphics[width=0.6\textwidth]{figuras/sensor_kinect.png}
\caption{Sensor Kinect}
\label{fig:sensor_kinect}
\end{figure}

Su funci�n principal es la de percibir el entorno captando una serie de puntos que se ubican en las tres dimensiones. Su funcionamiento a grandes rasgos se basa en un emisor de infrarrojos a 830 nm que interact�ua con los objetos y una c�mara infrarroja que etecta la diferencia entre la proyecci�n anterior y la actual, obteniendo la distancia a cada objeto.

En primer lugar, el laser infrarrojo es emitido por Kinect con un patr�n
determinado (Projected textures **REFERENCIA**), el cual no es sim�trico sino que tiene puntos aleatorios que se dispersa gracias a unas lentes de proyecci�n. Estos puntos aleatorios se reflejan en los objetos, los cuales ser�a posible verlos con una c�mara externa.

A continuaci�n, al sensor de Kinect MT9M001C12STM, que no es m�s que el
sensor CMOS de una c�mara en la que se le trata para que observe solo el
infrarrojo, obteniendo los puntos infrarrojos en el plano 2D. El motivo por el que podemos medir la profundidad de los objetos (su distancia) es porque sabemos el patr�n de c�mo emite el laser emisor \cite{konolige2010projected}, por tanto sabremos que si un punto no est� en el sitio que corresponde, se ha trasladado respecto al punto inicial y se le aplica la correspondiente transformaci�n (Figura \ref{fig:proyecciones_kinect}), obteniendo finalmente los puntos de toda la nube en coordenadas cartesianas XYZ.\\

\begin{figure}[htp]
\centering
\includegraphics[width=0.8\textwidth]{figuras/proyecciones_kinect.png}
\caption{Proyecci�n de infrarrojos y obtenci�n de la nube de puntos}
\label{fig:proyecciones_kinect}
\end{figure}

La siguiente tabla (Tabla \ref{tabla_kinect}) muestra las especificaciones del sensor Kinect.

\begin{table}[!h]
\centering
\resizebox{\textwidth}{!}{
\begin{tabular}{c c}
\hline
{\bf Especificaciones} & {\bf Sensor Kinect} \\ \hline
Dimensiones del conjunto & 270mm x 50mm x 70mm \\
Fuente infrarroja & 830nm \\
Potencia & 60 mW \\
C�mara Infrarroja & MT9M001C12STM \\
Resoluci�n c�mara infrarroja & 1200x960 pixeles \\
Frecuencia & 30 Hz \\
Tama�o pixel & 5.2um x 5.2um  \\
Pixeles activos & 1280H x 1024V \\
Campo de visi�n & 58� H, 45� V, 70� D \\
Resoluci�n espacial & 3mm (a 2 metros de distancia) \\
Resoluci�n de profundidad & 1cm (a 2 metros de distancia) \\
Distancia de operaci�n & 0.45m ? 6.5m \\
C�mara RGB & MT9M112 \\
Resoluci�n c�mara RGB & 640 x 480) \\
Audio & TAS1020B (Controlador de Audio) \\
Formato & 16kHz, 16-bit mono, modulaci�n por
codificaci�n de pulso (PCM)\\
Entrada de audio & 4 micr�fonos con conversi�n anal�gico
digital de 24bits \\
Aceler�metro & KXSD9-2050\\ \hline
\end{tabular}
}
\caption{Caracter�sticas del sensor Kinect}
\label{tabla_kinect}
\end{table}

\subsection{L�ser SICK LMS100}

Aunque el planteamiento incial del proyecto planteaba la navegaci�n basada �nicamente en el sensor kinect, debemos mencionar el uso del sensor l�ser Sick LMS100 (Figura \ref{fig:sensor_sicklms100}).

\begin{figure}[htp]
\centering
\includegraphics[width=0.6\textwidth]{figuras/SICK_LMS100.jpg}
\caption{Sensor escaner l�ser Sick LMS100}
\label{fig:sensor_sicklms100}
\end{figure}

Este es un sensor l�ser por infrarrojos de clase I (Inofensivo para el ojo humano), que obtiene la medida de distancias con gran preci�n y rapidez en un solo plano y realizando un barrido de 270� (Figura \ref{fig:sicklms100_rango}).

\begin{figure}[htp]
\centering
\includegraphics[width=0.6\textwidth]{figuras/sicklms100_rango.png}
\caption{Campo de visi�n del sensor l�ser Sick LMS100} \label{fig:sicklms100_rango}
\end{figure}

Este sensor est� colocado en la parte trasera del robot, enfocando hacia atr�s para cubrir un mayor rango y conocer todo el entorno alrededor del robot.

En la siguiente tabla (Tabla \ref{tabla_sicklms100}) se recogen sus caracter�sticas principales.

\begin{table}[!h]
\centering
\resizebox{\textwidth}{!}{
\begin{tabular}{c c}
\hline
{\bf Especificaciones} & {\bf Sick LMS100} \\ \hline
Campo de aplicaci�n & Interno \\
Fuente infrarroja & 905 nm \\
Clase L�ser & 1 (IEC 60825-1) \\
Campo de visi�n & 270� \\
Frecuencia de escaneo & 25Hz/50Hz \\
Resoluci�n angular & 0.25�/0.5� \\
Distancia de operaci�n & 0.05 - 20 m  \\
Tiempo de respuesta & 20 ms \\
Error & 30 mm \\
Interfaz de datos & Ethernet \\
Tensi�n de operaci�n & 10.8V - 20V DC \\
Consumo & 20 W \\
Peso & 1.1 Kg \\
Dimensiones & 105mm x 102mm x 152mm\\ \hline
\end{tabular}
}
\caption{Caracter�sticas del sensor l�ser Sick LMS100. Basado en \cite{sicklms100}}
\label{tabla_sicklms100}
\end{table}

\subsection{Intel NUC NUC5i7RYH}

El ordenador Intel NUC NUC5i7RYH, es un ordenador de altas prestaciones y de tama�o compacto que ofrece unas buenas caracter�sticas para procesar datos y realizar la algoritmia adecuada para tareas de rob�tica.

Est� equipado con un procesador Intel i7-5557U de quinta generaci�n que ofrece una frecuencai de reloj de 3.1 GHz. Est� incorporado con un discoduro de estado s�lido que permite una alta velocidad de lectura y escritura en disco, as� como una tarjeta RAM de tipo DDR3L de 8GB que permitir� el intercambio de informaci�n entre los nodos ROS de una manera fluida.

Su cometido ser� el de procesar la informaci�n de los sensores, generar los mapas incorporando los obst�culos, generar las trayectorias de navegaci�n y comandar los motores del robot para realizar movimientos.

Dispone de tama�o compacto y un consumo bajo, juto con una alimentaci�n a partir de los 12 voltios, lo que lo hace ideal para incorporarlo en robots m�viles que requieran realizar tareas sin depender de una infraestructura.

En la tabla \ref{tabla_intelnuc} pueden consultarse sus caracter�sticas principales.

\begin{table}[!h]
\centering
\resizebox{\textwidth}{!}{
\begin{tabular}{c c}
\hline
{\bf Especificaciones} & {\bf Intel NUC NUC5i7RYH} \\ \hline
Procesador & Intel Core i7-5557U, dual-core \\
Frecuencia de reloj & 3.1 GHz hasta 3.4 GHz \\
Memoria RAM & DDR3L1 **DATO** \\
Disco duro & M.2 SSD **DATO** \\
Gr�ficos & Iris Graphics 6100 \\
Conectividad de perif�ricos & 2 x USB 3.0 en el panel posterior\\
 & 2 x USB 3.0 en el panel frontal\\
 & 2 x USB 2.0 internos v�a colector\\
Conectividad de red & Intel 10/100/1000 Mbps\\
 & Intel� Wireless-AC 7265 M.2, antenas inal�mbricas (IEEE 802.11ac)\\
Alimentaci�n & 12-19V DC \\
Consumo & 65 W \\
Dimensiones & 115mm x 111mm x 48.7mm\\ \hline
\end{tabular}
}
\caption{Caracter�sticas del ordenador Intel NUC NUC5i7RYH}
\label{tabla_intelnuc}
\end{table}

\chapter{Arquitectura}
Esta secci�n tiene como objetivo plantear la arquitectura general utilizada en el robot, las comunicaciones con el resto del hardware y con los nodos que proporcionan la informaci�n necesaria para que el robot sea totalmente aut�nomo.

\section{Arquitectura general}

A nivel de hardware utilizado en el proyecto, como es el propio robot, los sensores el ordenador de abordo el sistema se estructura de a siguiente manera:

\begin{enumerate}[i)]
\item  Robot Pioneer 3 AT: Este es el robot mencionado anteriormente, el cual debe ser configurado para acceder al puerto serie RS-232 de su placa controladora. Esto nos permite conectarnos con el firmware ARCOS **referencia** y comunicarnos a trav�s de la librer�a ARIA. De esta forma controlamos los motroes y podemos leer el valor de los encoders de la odometr�a.

\item Sensores: Tanto el sensor Kinect como el sensor l�ser ir�n alimentados a trav�s de las bater�as del robot y se comunicar�n con el ordenador de abordo a trav�s de puerto USB y ethernet respectivamente.

\item Ordenador Intel NUC: Ser� el ordenador de abordo encargado de ejecutar ROS y realizar todo el procesamiento necesario. Ir� equipado con el sistema operativo Ubuntu 14.04 por ser la �ltima versi�n estable disponible a fecha de la entrega del proyecto. Ir� conectado al robot mediante un convertidor RS-232 a USB, el sensor l�ser se comunica v�a ethernet y el sensor Kinect a trav�s de puerto USB igualmente. Tambi�n se conectar� el audio al altavoz integrado del robot.

\item Ordenador externo: Como se ha mencionado anteriormente, un ordenador externo opcional equipado con ROS podr� utilizarse para realizar tareas de a supervisi�n inal�mrica a trav�s de RViz y para realizar la teleoperaci�n del robot v�a TCP/IP integrado en ROS.
\end{enumerate}

\section{Arquitectura del proyecto}

El proyecto est� estructurado siguiendo la filosof�a de "paquetes" desarrollados en ROS. En el directorio ra�z del proyecto por tanto, encontraremos los paquetes necesarios para que el sistema funcione as� como los paquetes propios desarrollados:




\chapter{Entorno ROS}

La versi�n del software ROS utilizada para el desarrollo del proyecto ha tratado de ser siempre la m�s actual posible, ya que eso nos asegura mantener la compatibilidad en futuros trabajos y que el software est� actualizado.

La versi�n utilizada fue ROS Indigo Igloo desde el comienzo del proyecto, bajo el sistema operativo Ubuntu 14.04. A fecha de entrega del proyecto existe una versi�n m�s actualizada del software ROS, sin embargo se desestim� su uso debido a que a�n no era una versi�n estable y algunos paquetes no se encontraban disponibles para tal versi�n.

ROS realiza funciones similares a un sistema operativo, como la comunicaci�n e interacci�n entre diferentes procesos, la dis tribuci�n en hilos, comunicaci�n distribuida entre m�quinas...

El concepto m�s importante dentro de ROS son los nodos, que no son m�s que un proceso que se ejecuta y conecta al proceso principal, llamado m�ster, para comunicarse con otros nodos. Existen diferentes conceptos rostopics, rosservices, nodelets... que ser�n explicados en el siguiente punto.

\subsection{Funcionamiento de ROS}

- Nodos.
- Servicios.
- Rostopics.

\section{Configuraci�n de ROS}
Para la instalaci�n de ROS es necesario seguir ciertos pasos bien explicados en la wiki de su p�gina.

Para el ordenador de abordo del robot utilizamos la versi�n completa del software.

A�adimos nuestra variable de entorno para buscar en los directorios de ros cuando ejecutemos los nodos.

Para el desarrollo dentro de ROS se utiliza el entorno de desarrollo Catkin, que facilita el enlazado y compilaci�n de los paquetes. para ello es necesario tenerlo instalado:

**instalacion catkin**

Y a continuaci�n es necesario incluir nuestro directorio de desarrollo para que sea reconocido:

**source**

A partir de este punto podr�amos realizar nodos utilizando las funcionalidades de ROS.

Para el desarrollo de este proyecto es necesario instalar las siguientes funcionalidades adicionales de ROS:

- Navegacion
- Rosaria
- pocketsphinx
- rosdeps
- mirar cuaderno

\section{Configuraci�n de los paquetes ROS}

Los paquetes ROS son funcionalidades desarrolladas por terceros que se integran en el sistema ROS y que son transferibles de un robot a otro.

Los paquetes ROS incorporan un archivo CMakeLists.txt para la compilaci�n de los nodos que se hayan desarrollado as� como sus mensajes. Y un archivo Package Manifest package.xml, donde se indican los requisitos del paquete, el autor, el contacto, y las dependencias del mismo.


\chapter{Dise�o del sistema}

\section{Jerarquizaci�n de tareas}

\section{Control de acciones mediante comandos de voz}

\section{Ejecuci�n aut�noma de los nodos}


\chapter{Implementaci�n del sistema}\label{chapter:implementacion}

Este cap�tulo describe los cambios, ajustes y modificaciones que, basados en la informaci�n anterior expuesta, las caracter�sticas de ROS y el hardware del que disponemos, se han realizado para alcanzar los objetivos del proyecto.

\section{Configuraciones hardware}
Como ya se ha descrito, la navegaci�n se basa en el sensor Kinect pero tambi�n se ha considerado integrar el sensor l�ser debido a la valiosa informaci�n que aporta y su disponibilidad.

Por ello, el robot deber� llevar incorporados estos sensores proporcion�ndoles alimentaci�n y una interfaz de conexi�n adecuada.

\subsection{Pioneer 3 AT}
El robot Pioneer 3 AT de Adept Mobile Robots es la base de la plataforma rob�tica. El modelo disponible en el laborarotio de la Escuela T�cnica Superior de Ingenier�a y Dise�o Industrial llevaba incorporado un ordenador de tipo **ORDENADOR PIONEER**. Tambi�n, al comienzo de este proyecto ya exist�an algunas adaptaciones como la incorporaci�n de un altavoz frontal, acceso a los puestos USB del ordenador interno y conexi�n para el sensor l�ser.

IMAGEN ROBOT ESTADO INICIAL.

El robot hab�a sido utilizado anteriormente mediante el software MRCore en el proyecto **PROYECTO DE ALEJANDRO** y el sistema operativo del ordenador interno era Ubuntu Server 10.

Para integrar la versi�n Indigo de ROS lo m�s recomendable era partir de la versi�n estable m�s actualizada de Ubuntu, por lo que se sustituy� el sistema operativo por Ubuntu 14.04 LTS en su versi�n de escritorio.

Una vez integrado el sistema operativo, la primera toma de contacto con el robot fue a partir de la librer�a Aria **referencia** para controlar el movimiento de los motores y comprobar que el robot se encontraba en buen estado.

**A�adir el puerto al grupo de dialout**

A continuaci�n, tras instalar ROS Indigo, se procedi� a las pruebas mediante el paquete Rosaria de ROS. La conexi�n con el microcontrolador de la placa de motores fue exitosa y se comprob� que los valores de la odometr�a tambi�n funcionaban.

**cambiar el puerto de conexion*?*


Llegados a este punto, ya dispon�amos del robot preparado para realizar las primeras pruebas.

\subsection{Sensor L�ser}
El sensor l�ser Sick ya hab�a sido integrado en un proyecto anterior y sus conexiones de alimentaci�n y datos v�a Ethernet ya estaban preparadas para utilizarlo.

Para conectarlo a trav�s del puerto Ethernet fue necesario ajustar su direcci�n IP a trav�s del software del fabricante y ajustar la IP del ordenador del robot Pioneer (m�s informaci�n en el ap�ndice **TAL**).

El agarre mec�nico del sensor se dej� tal y como hab�a sido utilizado en ocasiones anteriores, situado en la parte frontal agarrado mediante un par de tornillos al chasis con tuercas de palometa para su f�cil manipulaci�n.

**Imagen del sensor l�ser en el robot**.

El sensor l�ser se conecta a la interfaz ROS mediante el paquete LMS1xx tal y como se describi� en el apartado \ref{subsection:sicklms100}.

\subsection{Sensor Kinect}
La integraci�n del sensor Kinect fue relativamente sencilla debido a que las entradas de los puestos USB del ordenador hab�an sido cableadas previamente. La adaptaci�n a realizar era sobre la parte de alimentaci�n, ya que este sensor trabaja a una tensi�n de 12 voltios.

En el manual del robot se encuentra una descripci�n detallada de la placa de alimentaci�n a la cual pueden conectarse diferentes perif�ricos. Esta placa ofrece tomas de conexi�n de 5 voltios controlados por unos botones auxiliares y tomas de 12 voltios (ver ap�ndice **TAL**).

El sensor Kinect dispone de un adaptador USB, preparado para trabajar con la videoconsola XBOX 360, el cual suministra 12 voltios mediante un transformador conectado a una toma de corriente alterna de 220v e incorpora los cables de datos del propio sensor Kinect.

**Esquema conexion usual**.

Para integrar el sensor Kinect en el robot, se cort� el cable de alimentaci�n del cable adaptador y se soldaron unas clavijas tipo Jack **REVISAR** macho-hembra para conectar el adaptador directamente a los 12 voltios de la placa del robot. Tambi�n se realiza� lo oportuno en el adaptador de corriente, para poder usar el sensor Kinect de la manera habitual.

\begin{figure}[!htp]
\centering
\includegraphics[width=\textwidth]{figuras/adaptador_kinect.png}
\caption{Adaptaci�n de cables para la alimentaci�n del sensor Kinect}
\label{fig:cables_kinect}
\end{figure}

Para anclar el sensor Kinect al robot se opt� por situarlo en la parte superior del sensor L�ser, para lo cual se dise�� una pieza que encajase en la base de la Kinect y en el sensor l�ser (figura **REF**).

**Imagen del dise�o 3D**

El sensor Kinect se conecta a la interfaz ROS mediante el paquete freenect\_stack tal y como se describi� en el apartado \ref{subsection:kinect}.

\subsection{Primera configuraci�n hardware}
La primera configuraci�n del robot consisti� en ambos sensores situados en la parte frontal del mismo. Los sensores de encontraban colocados de manera vertical otro, de tal forma que no existieran interferencias entre uno y otro.

\begin{figure}[!htp]
\centering
\includegraphics[width=\textwidth]{figuras/primera_configuracion.jpg}
\caption{Primera configuraci�n hardware del robot}
\label{fig:primera_configuracion}
\end{figure}

De esta forma consegu�amos una vista frontal despejada y cont�bamos con la informaci�n del l�ser para detectar obst�culos laterales.

\subsubsection{Primera configuraci�n del sistema}
El ordenador interno corr�a todos los nodos de ROS, de modo que se dispon�a de la informaci�n de los sensores, el control sobre los motores y la lectura de la odometr�a para realizar las primeras pruebas con el paquete de navegaci�n de ROS (Secci�n \label{section:navigation_stack}).

Sin embargo, la primera implementaci�n con los primeros ajustes a nuestro hardware de los sensores no fue posible debido a la sobrecarga de la CPU del ordenador interno del robot Pioneer y a problemas de memoria en la ejecuci�n de nodos como AMCL.

\subsubsection{Segunda configuraci�n del sistema}
La siguiente opci�n fue utilizar un ordenador externo que realizase los c�lculos de navegaci�n y enviase al robot las consignas de movimiento a trav�s de una red WLAN. Esta idea no era la soluci�n m�s ideal, ya que desde el principio la idea era que el robot fuese lo m�s aut�nomo posible sin depender de una infraestructura, sin embargo era una posibilidad directa que no supon�a mucho esfuerzo.

**esquema configuraci�n**

Gracias a la filosof�a de ejecuci�n distribuida de nodos de ROS, ejecutar nodos en m�quinas diferentes y compartir la informaci�n entre procesos es una tarea sencilla. Para configurarlo, tan solo es necesario indicar a las m�quinas la IP del nodo master. De esta forma, los nodos que se ejecuten en cada una de las m�quinas tratar�n de realizar la comunicaci�n a trav�s de IPs dentro de la misma red.

Este ajuste fue puesto en marcha utilizando un ordenador port�til con suficiente capacidad de procesamiento y memoria como para ejecutar la navegaci�n, sin embargo aparecieron algunos inconvenientes.

El primero de ellos fueron las direcciones IP en el ordenador interno del Pioneer.

\textbf{Problemas con el sensor L�ser: LM1xx}\\
Debido a que el sensor l�ser se conecta v�a Ethernet a este ordenador, el nodo LMS1xx debe obtener informaci�n a trav�s de la IP del l�ser y enviarla a trav�s del adaptador Wifi a la IP del nodo m�ster. El problema resid�a en que el nodo se saturaba al tener que lidiar con ambas interfaces de conexi�n y provocaba su detenci�n.

Tras varias consultas a Clearpath Robotics a trav�s de su repositorio de GitHub y preguntas en el foro ROS Answers **Enlaces de referencia**, la soluci�n no estaba implementada en c�digo y lo m�s inmediato era hacer un bridge en el ordenador del Pioneer 3 AT entre la interfaz Ethernet y la Wifi.

Los resultados de esta soluci�n no fueron satisfactorios ya que el comportamiento era el mismo: el nodo LMS1xx se saturaba e interrump�a a los pocos minutos de su ejecuci�n.

Trantando de resolver este problema, se hicieron pruebas generando una red Wifi Ad-hoc desde el ordenador del robot, a la cual se conectaba el ordenador externo. Los resultados fueron buenos siempre y cuando las IPs del nodo master y del sensor L�ser se encontrasen en el mismo subrango **revisar nomenclatura**.

En la implementaci�n final del sistema esta soluci�n se sigue utilizando para conectarse desde un ordenador externo al ordenador que incorpora el robot Pioneer.

**OJOOOO continua seccion**

Una vez se pudo conectar el ordenador externo, la ejecuci�n del nodo de navegaci�n era la correcta y las consignas de movimiento se enviaban correctamente al robot, pudiendo realizar las primeras pruebas de navegaci�n aut�noma.

**Imagen primeras navegaciones**

Sin embargo en ocasiones la recepci�n y env�o de datos era demasiado alta y esto provocaba que existiese mucho retraso en la comunicaci�n, haciendo que el robot reaccionase tarde para esquivar los obst�culos y el control del robot fuera impracticable.

\subsubsection{Tercera configuraci�n del sistema}
Finalmente se opt� por montar un ordenador m�s potente en el robot, para lo cual se utiliz� un port�til externo al que se conectaba tanto e sensor Kinect como el sensor L�ser y se utilizaba un convertidor de puesto USB a puerto serie RS-232 para controlar el movimiento del robot y leer la odometr�a. El ordenador del robot quedaba sustituido y as� se mantuvo hasta la versi�n final.

**imagen del robot con el port�til**

Llegados a este punto, ahora s� dispon�amos de la plataforma rob�tica completa sobre la que trabajar en la navegaci�n del robot. Las primeras pruebas fueron satisfactorias, logrando correr todos los nodos en el port�til, el cual se incorpor� de manera provisional al robot por medio de unos agarres realizados con una impresora 3D.

**imagen de los agarres**

\subsection{Segunda configuraci�n}

La segunda configuraci�n hardware vino dada tras las pruebas satisfactorias con l
\subsubsection{Configuraci�n del sistema final}
\textbf{Intel NUC}

\section{Navegaci�n}
\subsection{Sensores}
\subsection{Navegaci�n con mapa}
\subsection{Navegaci�n reactiva}
\section{Nodo de navegaci�n por puntos}
\section{Nodo de comandos por voz}
\subsection{Reconocimiento de comandos de voz}
\subsection{Feedback mediante text-to-speech}

\section{Nodo de ejecuci�n autom�tica de nodos}


\chapter{Pruebas del sistema}

En este cap�tulo...

\section{Simulaci�n}
Simulaci�n con MobileSim
Simulaci�n con Gazebo

\section{Visualizaci�n mediante RViz}


\section{Pruebas reales}

\include{capitulos/conclusion}

%genera doble hoja en blanco
\cleardoublepage
%se incluye la bibliograf�a. Archivo de tipo .bib (bibtex)
%\bibliographystyle{alpha}
%\bibliography{bibliografia/bibliografia}
\printbibliography

\appendix
\addappheadtotoc
\appendixpage
\chapter{Configuraci�n del sistema} \label{chapter:configuracion_sistema}

En este ap�ndice se resume de manera simplificada y muy enfocada la configuraci�n completa del sistema rob�tico objeto de este trabajo, tanto la parte referente al software de robot como la parte destinada a la configuraci�n hardware de todos los equipos que incorpora.

\section{Configuraci�n del espacio de trabajo}
En ROS el espacio de trabajo es el lugar donde se realiza el desarrollo software de los paquetes ROS. Este en torno de trabajo es capaz de gestionar la correcta compilaci�n de los nodos.

El espacio de trabajo es el determinado por la herramienta Catkin. A partir de la versi�n Indigo de ROS, casi todos los paquetes se encuentran adaptados al entorno Catkin y funcionan de manera casi inmediata.

A continuaci�n se describen los pasos para crear un espacio de trabajo Catkin (extra�dos del tutorial *TAL*) y los pasos para clonar el repositorio de GitHub que aloja el c�digo.

Tras instalar ROS en el equipo deseado, instalamos Catkin:

\begin{code}[htp]
\begin{lstlisting}[style=consola]
$ sudo apt-get install ros-indigo-catkin
$ mkdir -p ~/catkin_ws/src
$ cd ~/catkin_ws/src
$ catkin_init_workspace
\end{lstlisting}
\caption{Instalaci�n y workspace de Catkin}
\end{code}

A continuaci�n inncluimos el directorio de desarrollo para que sea reconocido por ROS a la hora de buscar dependencias:

\begin{code}[!htp]
\begin{lstlisting}[style=consola]
$ cd ~/catkin_ws
$ source devel/setup.bash
\end{lstlisting}
\caption{Source al setup de nuestro entorno Catkin}
\end{code}

\subsection{Instalaci�n de las librer�as}

Para clonar el desarrollo software de este proyecto bastar�a con clonar el repositorio de GitHUb en el que se ha trabajado en este proyecto \url{https://github.com/danimtb/pioneer3at_ETSIDI} en nuestra carpeta \textit{$\sim$/catkin\_ws/src}.

\begin{code}[htp]
\begin{lstlisting}[style=consola]
$ cd ~/catkin_ws/src
$ $ git clone --recursive https://github.com/danimtb/pioneer3at_ETSIDI.git .
\end{lstlisting}
\hypersetup{urlcolor=black}
Fuente: \href{https://github.com/danimtb/pioneer3at_ETSIDI}{\textit{https://github.com/danimtb/pioneer3at\_ETSIDI}}
\hypersetup{urlcolor=blue}
\caption{Clonado del repositorio \textit{pioneer3at\_ETSIDI}}
\end{code}

Sin embargo, es recomendable que si se desea realizar alg�n desarrollo posterior, se realice un fork en GitHub de este proyecto (Figura *TAL*) y se clone el repositorio propio.

**Captura de pantalla del repo github**

De esta forma podemos guardar los cambios realizados en el repositorio de la persona que hace la modificaci�n, con la intenci�n de incorporar los cambios m�s tarde al repositorio de desarrollo principal (\url{https://github.com/danimtb/pioneer3at_ETSIDI.git}) mediante Pull Request.

**figura pull request**


\subsection{Gesti�n de las dependencias}

 ROS se vale de la informaci�n almacenada en la definici�n de cada uno de sus paquetes en \textit{package.xml} y gestionar algunas dependencias de librer�as, para lo cual se sirve de la herramienta rosdep **referencia** creada por los desarrolladores de ROS para este prop�sito.

Para comenzar a utilizarla inicializamos rosdep y actualizamos las dependencias:

\begin{code}[htp]
\begin{lstlisting}[style=consola]
$ sudo rosdep init
$ rosdep update
\end{lstlisting}
\caption{Inicializando la herramienta \textit{rosdep}.}
\end{code}

A continuaci�n se describen los pasos necesarios para instalar las dependencias de cada secci�n:

\begin{itemize}
\item Navegaci�n: Instalamos los paquetes adicionales para realizar la navegaci�n y mapeo (Cap�tulo \ref{chapter:navegacion}).
\end{itemize}

\begin{code}[!htp]
\begin{lstlisting}[style=consola]
$ sudo apt-get install ros-indigo-navigation
$ sudo apt-get install ros-indigo-gmapping
\end{lstlisting}
\caption{Instalando los paquetes de navegaci�n mapeo.}
\end{code}

\begin{itemize}
\item Rosaria: Instalamos las dependencias de RosAria, principalmente la librer�a Aria.
\end{itemize}

\begin{code}[!htp]
\begin{lstlisting}[style=consola]
$ rosdep install rosaria
\end{lstlisting}
\caption{Instalando las dependencias de rosaria.}
\end{code}


Para el reconocimiento de los comandos de voz utilizamos el paquete \textit{pocketsphinx}:

\begin{itemize}
\item pocketsphinx: Instalamos los paquetes de conversion a texto y buscamos las dependencias del paquete.
\end{itemize}

\begin{code}[!htp]
\begin{lstlisting}[style=consola]
$ sudo apt-get install gstreamer0.10-gconf
$ rosdep install pocketsphinx
\end{lstlisting}
\caption{Instalando las dependencias de \textit{pocketsphinx} y \textit{gstreamer}.}
\end{code}

Es posible que necesitemos las dependencias a�adidar de \textit{audio\_capture}.

\begin{code}[!htp]
\begin{lstlisting}[style=consola]
$ rosdep install audio_capture
\end{lstlisting}
\caption{Instalando las dependencias de \textit{audio\_capture}.}
\end{code}

Para el audio (sintetizado de voz) utilizamos el nodo \textit{sound\_play}, por lo que debemos instalar sus dependencias.

\begin{itemize}
\item sound\_play: Nodo para el sintetizado de voz y reproducir sonidos.
\end{itemize}

\begin{code}[!htp]
\begin{lstlisting}[style=consola]
$ rosdep install sound_play
\end{lstlisting}
\caption{Instalando las dependencias de \textit{sound\_play}.}
\end{code}

Una vez disponemos de todos los paquetes y sus respectivas dependencias no debemos olvidarnos de que es necesario compilarlos:

\begin{code}[!htp]
\begin{lstlisting}[style=consola]
$ cd ~/catkin_ws
$ catkin_make
\end{lstlisting}
\caption{Compilando los paquetes del espacio de trabajo Catkin.}
\end{code}


\section{Configuraci�n del hardware}

En este ap�ndice se describen algunos ajustes �tiles del hardware que se ha utilizado en el proyecto. Esta informaci�n es importante para que el sistema rob�tico desarrollado funcione pero no ha sido incluida dentro del cuerpo de la memoria para no alargar las explicaciones.

En cualquier caso, los procemientos que a continuaci�n se describen es probable que no sea necesarioz volver a realizarlos en futuros usos del robot si se mantiene la configuraci�n descrita en este trabajo.

\subsection{Calibraci�n de la odometr�a}\label{subsection:odometria}
La lectura de la posici�n de los motores del robot se realiza mediante unos encoders situados en el eje de cada motor.

El firmware ARCOS de la placa controladora del robot es actualizable y gestiona os par�metros de calibraci�n de la odometr�a. Este firmware es actualizable aunque para ello es necesario ponerse en contacto con el fabricante Adept.

Estos par�metros referentes a la odometr�a son configurables a trav�s de la librer�a Aria en caso de que no se desee acceder a modificar el firmware del robot.

Los par�metros a considerar son los siguientes:

\begin{itemize}
\item TicksMM
\item DriftFactor
\item RevCount
\end{itemize}

El nodo de ROS RosAria es capaz de pasar estos par�metros a la placa controladora del robot a trav�s de Aria. Es por ello que podemos configurar din�micamente estos par�metros del robot cada vez que ejecutamos el nodo. De esta forma evitamos tener que recurrir a Adept y a software espec�fico para modificar los par�metros de ARCOS.

Podemos ver la descripci�n de estos par�metros en el manual del robot Pioneer 3 AT **referencia**

**IMAGEN DEL MANUAL**

EL procedimiento para calcular cada uno de estos par�metros es el siguiente:



\subsection{Ordenador de abordo Intel NUC}

\subsection{Sensor Kinect}
El sensor Kinect utilizado en el proyecto no dispone de ninguna moficiaci�n espec�fica o modo de funcionamiento especial. Lo referente a su conexionado ha sido descrito anteriormente en la secci�n \label{subsection:implementacion_kinect}.

Tan solo indicar que puede realizarse una primera toma de contacto a trav�s de la librer�a \textit{libfreenect}. Una manera r�pida de hacerlo es utilizando el gestor de depencencias en C++ biicode **LINK A BIICODE**, en un sistema operativo Ubuntu siguiendo esta gu�a **URL DE BIICODE**.

\subsection{L�ser SICK LMS100}\label{subsection:sicklms100_apendice}

El sensor l�ser LMS100 de la marca Sick es un sensor de ampli rango y largo alcance de tipo industrial.

El fabricante Sick ofrece un software propietario llamado SOPAS Ingineering Tool que permite acceder a la configuraci�n interna del sensor. Este software tan solo puede utilizarse bajo el sistema operativo Microsoft Windows, en concreto ha sido utilizado con Windows 7.

Para instalarlo hay que recurrir a la web del fabricante: **COMPLETAR**

Para configurar el dispositivo debemos enchufarlo a un puerto Ethernet de un ordenador con el software instalado y proporcionarle alimentaci�n.

El puerto ethernet del ordenador debe estar configurado para obtener una IP autom�tica. Una vez aparezca la red como conectada procedemos a abrir el software SOPAS.

\begin{figure}[!htp]
\centering
\includegraphics[width=0.8\textwidth]{figuras/sopas1.png}
\caption{Software Sick SOPAS con el sensor detectado.}
\label{fig:sopas1}
\end{figure}

Autom�ticamente detectar� el dispositivo pero seguramente no podamos acceder a �l debido a una direcci�n IP err�nea.

Para ello abriremos el apartado de configuraci�n y selecionamos ''Asignar IP autom�ticamente" para que nos asigne el rango adecuado.

\begin{figure}[!htp]
\centering
\includegraphics[width=0.8\textwidth]{figuras/sopas2.png}
\caption{Configurando la direcci�n IP del dispositivo.}
\label{fig:sopas2}
\end{figure}

Si queremos una direcci�n IP de determinado rango, la mejor manera de proceder es asignar al adaptador de red del ordenador una IP manual del rango deseado y realizar el procedimiento anterior.

Tambi�n podemos acceder a los par�metros del sensor y visualizar en un gr�fico la informaci�n que est� captando en el momento.

\begin{figure}[!htp]
\centering
\includegraphics[width=0.8\textwidth]{figuras/sopas3.png}
\caption{Gr�fico de la informaci�n captada por el sensor en el software SOPAS.}
\label{fig:sopas3}
\end{figure}

A partir de aqu� podremos conectarnos al l�ser con el nodo ROS LMS1xx, configurando una IP fija en el adaptador de red ethernet del ordenador al que lo conectemos, en este caso el ordenador Intel NUC del robot.

\begin{itemize}
\item Direcci�n IP del l�ser Sick LMS100:
\item Direcci�n IP del adaptador ethernet del ordenador Intel NUC: 
\end{itemize}

Para m�s informaci�n conviene leer el manual que ofrece Sick \cite{sicklms100}.

\chapter{Manual de uso del robot}
Este ap�ndice es una gu�a de funcionamiento del robot Pioneer 3 AT (Petrois) utilizado e implementado al acabar este proyecto fin de grado.

\section{Encendido del robot}
El robot se enciende mediante un interruptor situado en su parte trasera, el cual proporciona alimentaci�n general a todos los sistemas.

Adicionalmente es necesario encender el ordenador de abordo Intel NUC mediante su correspondiente bot�n y el altavoz frontal del robot, para lo cual ser� necesario tener activada la alimentaci�n auxiliar 2 (AUX 2) en el panel de control.

**DESCRIPCION ENCENDIDO INTEL NUC**

\section{Panel de control y parada de emergencia}
El panel de control del robot se sit�a en su lateral derecho. Aqu� disponemos de una serie de botones, indicadores luminosos, ac�sticos y conexiones.

\begin{figure}[!htp]
\centering
\includegraphics[width=0.6\textwidth]{figuras/panel_derecho.jpg}
\caption{Panel de control del robot Petrois.}
\label{fig:panel_derecho}
\end{figure}

A continuaci�n se describe cada uno de los elementos de la figura \ref{fig:panel_derecho}:
\begin{itemize}
\item Pulsador MOTORS: Encargado de habilitar y deshabilitar los motores del robot. La activaci�n de los mismos puede realizarse a trav�s de software. Es necesario pulsar este bot�n para rearmar el robot cuando se pulsa la seta de emergencia. Si se pulsa una vez m�s el robot realiza una secuencia de movimientos para comprobar que los motores funcionan correctamente.
\item Pulsador RESET: Resetea la placa controladora del robot. El robot queda su estado inicial, tal y como si lo acab�semos de encender. Al pulsar este bot�n se interrumpe cualquier comunicaci�n con la placa de control.
\item Pulsador AUX 1: Habilita la alimentaci�n 1 de la placa de alimentaci�n. No se encuentra en uso.
\item Pulsador AUX 2: Habilita la alimentaci�n 2 de la placa de alimentaci�n. Es necesario tenerlo habilitado para que el altavoz frontal del robot funcione.
\item LED PWR: Indica el estado de los motores.
\item LED STATUS: Indica el estado del robot.
\item LED BATERY: Indica el estado de carga de la bater�a.
\item LEDs AUX 1 y AUX 2: Indica si las alimentaciones auxiliares se encuentran activas.
\item LEDs RX y TX: Muestran el estado de la comunicaci�n con la placa de control.
\item SERIAL: Puerto RS-232 que comunica con la placa de control. Puede ser utilizado para conectar un ordeandor externo directamente la placa de control del robot.
\end{itemize}

Muchos de los estados anteriores se indican mediante una se�al ac�stica. El reseteo de la placa tiene un sonido caracter�stico al igual que cuando se pierde la conexi�n con la placa de control.

El pulsador de emergencia se encuentra en la parte superior del robot y es de vital importancia cuando el robot se mueve de manera descontrolada, choca o est� en peligro la integridad de una persona o del propio robot. Es por ello que se recomienda su uso siempre que se encuentre en una situaci�n de las anteriores.

**imagen de la seta de emergencia**

Al pulsar la seta de emergencia �sta quedar� enclavada y el robot dar� una se�al ac�stica continuada indicando que se encuentra en parada. En este punto los motores quedan deshabilitados y las ruedas del robot pueden moverse libremente.

Para rearmar el robot basta con devolver la seta de emergencia a su posici�n original girando levemente la misma en el sentido de las flechas. Despu�s es necesario pulsar el bot�n MOTORS para volver a habilitar los motores.

Cabe indicar que la comunicaci�n de cualquier ordenador con la placa de control del robot no se interrumpe, por lo que al rearmar el robot es posible que �ste contin�e con los movimientos previos a la parada de emergencia. Aseg�rese de que las consignas de movimiento son las correctas y ponga especial precauci�n cuando realice el rearmado del robot.

Para informaci�n m�s extensa y detallada se recomienda leer la gu�a de usuario ofrecida por el fabricante **REFERENCIA**.

\section{Conexi�n mediante un ordenador externo v�a Wifi}

Al encender el ordenador Intel NUC, �ste est� configurado para generar directamente un HotSpot Wifi o red Ad-hoc con el nombre "P3AT". Esta red Wifi maneja direcciones IP en el rango 10.42.0.X por lo que es recomendable conectarse a ella con una IP fija que est� dentro de ese rango. SU contrase�a es "pioneer3at"

Si se est� ejecutando el nodo MASTER en el ordenador Intel NUC, �ste estar� configurado con la IP 10.42.0.1. Para indicar que queremos conectarnos a un nodo MASTER que se ejecuta en otra m�quina debemos editar el script \textit{.bashrc} del ordenador externo, que se ejecuta siempre que abrimos una terminal.

Para editarlo, abrimos la terminal y escribimos:

\begin{code}[htp]
\begin{lstlisting}[style=consola]
$ gedit ~/.bashrc
\end{lstlisting}
\caption{Abriendo el \textit{.bashrc}.}
\end{code}

Una vez abierto, al final del archivo a�adimos las siguientes dos l�neas:

\begin{code}[htp]
\begin{lstlisting}[style=consola]
export ROS_IP=10.42.0.77
export ROS_MASTER_URI=http://10.42.0.1:11311
\end{lstlisting}
\caption{A�adidendo las direcciones IP al \textit{.bashrc}.}
\end{code}

La primera l�nea indica la IP fija con la que el ordenador externo se conecta a la red Ad-hoc del Intel NUC. Modificar la IP y escribir la IP del ordeandor externo.
La direcci�n IP del MASTER es la direcci�n que est� configurada en el Intel NUC y a la que tratar�n de conectarse los nodos cuando se lancen desde el ordenador externo.

Este procedimiento est� hecho igual en el Intel NUC pero indicando tan solo que la IP de ese equipo es \textbf{ROS\_IP=10.42.0.77}.

\section{Acceder a la placa de alimentaci�n}

La placa de alimentaci�n se encuentra bajo la tapa negra principal del robot, justo debajo del sensor Kinect.

Su posici�n no es muy accesible y se recomienda encarecidamente leer el manual para tener acceso a la misma.

Por otro lado, siempre puede acderse a la alimetaci�n de 12 voltios a trav�s del cable de alimentaci�n del sensor Kinect y a 5 voltios a trav�s de los cables de alimentaci�n del altavoz delantero.

\section{Cargar las bater�as del robot}
El robot Pioneer 3 AT dispone de un pack de 3 bater�as est�ncas de plomo-�cido a 12 voltios. Estas bater�as suministran alimentaci�n a todos los sistemas del robot y proporcionan una autonom�a aproximada de 2 horas de funcionamiento.

El acceso a las bater�as se encuentra en el interior del chasis del robot y son accesibles mediante una trampilla trasera.

**IMAGEN**

La carga de las bater�as se realiza con el adaptador de la figura **TAL** a trav�s del conector de carga situado junto al interruptor de alimentaci�n general.

**Imagen del cargador**

No ha podido estimarse el tiempo de carga completo, sin embargo el propio cargador incorpora un indicador luminoso que muestra el estado de carga de las bater�as.

\section{Ordenador interno del Pioneer 3 AT}

EL ordenador interno del robot Pioneer 3 AT se encuentra en desuso.

Todos los controles necesarios para acceder al ordenador est�n disponibles a trav�s del panel situado en el lateral izquierdo del robot.

\begin{figure}[!htp]
\centering
\includegraphics[width=0.5\textwidth]{figuras/panel_izquierdo.png}
\caption{Panel del ordenador interno (lateral izquierdo) del robot Petrois.}
\label{fig:panel_izquierdo}
\end{figure}

En este lateral encontramos elementos adicionales como la antena de conexi�n wifi del ordenador interno y el acceso a los puertos USB y Jacks de micr�fono y auriculares.

\textbf{IMPORTANTE:}\\
Originalmente este ordenador se encontraba conectado a la placa controladora del robot de manera interna a trav�s el puerto COM1 (Accesible en el interior del robot). En su lugar se conect� el convertidor de RS-232 a USB utilizado con el ordeandor Intel NUC, por lo que el ordenador interno se encuentra desconectado del robot.

\textbf{IMPORTANTE:}\\
Uno de los ventiladores internos situado en la parte frontal derecha del robot fue desconectado debido a su elevado ruido y su encendido permanente. En caso de utilizar el ordenador interno del robot es posible que sea necesario volver a conectarlo. Sin embargo no existe riesgo de sobrecaentamiento ya que el propio ordenador dispone de un ventilador adicional que se pone en marcha al encencerlo.


\chapter{Informaci�n y documentos ONLINE}
 En este ap�ndice se muestra informaci�n adicional de todo tipo referente al proyecto y a los materiales utilizados.

\section{Repositorio de c�digo}
Como se ha descrito anteriormente, el repositorio de c�digo desarrollado en ente proyecto se encuentra almacenado en GitHub. Puede consultarse su linea de tiempos, commits e incluso realizar preguntas en el apartado de Issues.

\url{https://github.com/danimtb/pioneer3at_ETSIDI}

**Im�gen del repositorio**

En �l podr� encontrar una peque�a gu�a \textit{Readme} actualizada con informaci�n sobre las utilidades que ofrece el repositorio y en concreto el desarrollo realizado en el paquete \textit{pioneer\_utils}, donde se encuentra la mayor�a del desarrollo.

\subsection{Readme}
Copia del documento \textit{Readme} del repositiorio.

**COPIA**

\section{Preguntas en ROS Answers y Github}
Pregunta del l�ser
Preguntal del Intel Nuc

Issue cleapath robotics

\section{Multimedia}
Colecci�n de v�deos:

\begin{itemize}
\item V�deos de este proyecto:
\item V�deos de antiguos proyectos:
\end{itemize}

Im�genes:
\begin{itemize}
\item Im�genes de este proyecto:
\item Im�genes de antiguos proyectos:
\end{itemize}

\section{Memoria del trabajo}

La memoria del trabajo ha sido desarrollada en \LaTeX, bajo el editor \textit{TexStudio} en un ordenador con sistema operativo GNU/Linux Ubuntu 14.04 LTS.

Su desarrollo ha sido puesto bajo un controlador de versiones y alojado en el siguiente repositorio de github:

\url{https://github.com/danimtb/TFG_pioneer3at}

En el propio repositorio se encuentran las figuras utilizadas y los archivos .tex, donde se ha escrito el mismo,  y la bibligraf�a empleada.

El documento PDF compilado de \LaTeX puede consultarse en el siguiente enlace:

\url{https://github.com/danimtb/TFG_pioneer3at/TFG.pdf}

\backmatter

%fin del documento
\end{document}
