\chapter{Arquitectura}
Esta secci�n tiene como objetivo plantear la arquitectura general utilizada en el robot, las comunicaciones con el resto del hardware y con los nodos que proporcionan la informaci�n necesaria para que el robot sea totalmente aut�nomo.

\section{Arquitectura general}

A nivel de hardware utilizado en el proyecto, como es el propio robot, los sensores el ordenador de abordo el sistema se estructura de a siguiente manera:

\begin{enumerate}[i)]
\item  Robot Pioneer 3 AT: Este es el robot mencionado anteriormente, el cual debe ser configurado para acceder al puerto serie RS-232 de su placa controladora. Esto nos permite conectarnos con el firmware ARCOS **referencia** y comunicarnos a trav�s de la librer�a ARIA. De esta forma controlamos los motroes y podemos leer el valor de los encoders de la odometr�a.

\item Sensores: Tanto el sensor Kinect como el sensor l�ser ir�n alimentados a trav�s de las bater�as del robot y se comunicar�n con el ordenador de abordo a trav�s de puerto USB y ethernet respectivamente.

\item Ordenador Intel NUC: Ser� el ordenador de abordo encargado de ejecutar ROS y realizar todo el procesamiento necesario. Ir� equipado con el sistema operativo Ubuntu 14.04 por ser la �ltima versi�n estable disponible a fecha de la entrega del proyecto. Ir� conectado al robot mediante un convertidor RS-232 a USB, el sensor l�ser se comunica v�a ethernet y el sensor Kinect a trav�s de puerto USB igualmente. Tambi�n se conectar� el audio al altavoz integrado del robot.

\item Ordenador externo: Como se ha mencionado anteriormente, un ordenador externo opcional equipado con ROS podr� utilizarse para realizar tareas de a supervisi�n inal�mrica a trav�s de RViz y para realizar la teleoperaci�n del robot v�a TCP/IP integrado en ROS.
\end{enumerate}

\section{Arquitectura del proyecto}

El proyecto est� estructurado siguiendo la filosof�a de "paquetes" desarrollados en ROS. En el directorio ra�z del proyecto por tanto, encontraremos los paquetes necesarios para que el sistema funcione as� como los paquetes propios desarrollados:


