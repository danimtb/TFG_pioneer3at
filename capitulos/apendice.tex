\appendix
\addappheadtotoc
\appendixpage
\chapter{Configuraci�n del sistema} \label{chapter:configuracion_sistema}

En este ap�ndice se resume de manera simplificada y muy enfocada la configuraci�n completa del sistema rob�tico objeto de este trabajo.

\section{Configuraci�n del espacio de trabajo}
En ROS el espacio de trabajo es el lugar donde se realiza el desarrollo software de los paquetes ROS. Este en torno de trabajo es capaz de gestionar la correcta compilaci�n de los nodos.

El espacio de trabajo es el determinado por la herramienta Catkin. A partir de la versi�n Indigo de ROS, casi todos los paquetes se encuentran adaptados al entorno Catkin y funcionan de manera casi inmediata.

A continuaci�n se describen los pasos para crear un espacio de trabajo Catkin (extra�dos del tutorial *TAL*) y los pasos para clonar el repositorio de GitHub que aloja el c�digo.

Tras instalar ROS en el equipo deseado, instalamos Catkin:

\begin{code}[htp]
\begin{lstlisting}[style=consola]
$ sudo apt-get install ros-indigo-catkin
$ mkdir -p ~/catkin_ws/src
$ cd ~/catkin_ws/src
$ catkin_init_workspace
\end{lstlisting}
\caption{Instalaci�n y workspace de Catkin}
\end{code}

A continuaci�n inncluimos el directorio de desarrollo para que sea reconocido por ROS a la hora de buscar dependencias:

\begin{code}[!htp]
\begin{lstlisting}[style=consola]
$ cd ~/catkin_ws
$ source devel/setup.bash
\end{lstlisting}
\caption{Source al setup de nuestro entorno Catkin}
\end{code}

\subsection{Instalaci�n de las librer�as}

Para clonar el desarrollo software de este proyecto bastar�a con clonar el repositorio de GitHUb en el que se ha trabajado en este proyecto \url{https://github.com/danimtb/pioneer3at_ETSIDI} en nuestra carpeta \textit{$\sim$/catkin\_ws/src}.

\begin{code}[htp]
\begin{lstlisting}[style=consola]
$ cd ~/catkin_ws/src
$ $ git clone --recursive https://github.com/danimtb/pioneer3at_ETSIDI.git .
\end{lstlisting}
\hypersetup{urlcolor=black}
Fuente: \href{https://github.com/danimtb/pioneer3at_ETSIDI}{\textit{https://github.com/danimtb/pioneer3at\_ETSIDI}}
\hypersetup{urlcolor=blue}
\caption{Clonado del repositorio \textit{pioneer3at\_ETSIDI}}
\end{code}

Sin embargo, es recomendable que si se desea realizar alg�n desarrollo posterior, se realice un fork en GitHub de este proyecto (Figura *TAL*) y se clone el repositorio propio.

**Captura de pantalla del repo github**

De esta forma podemos guardar los cambios realizados en el repositorio de la persona que hace la modificaci�n, con la intenci�n de incorporar los cambios m�s tarde al repositorio de desarrollo principal (\hypersetup{urlcolor=black} \href{https://github.com/danimtb/pioneer3at_ETSIDI}{\textit{https://github.com/danimtb/pioneer3at\_ETSIDI}}\hypersetup{urlcolor=blue}
) mediante Pull Request.

**figura pull request**


\subsection{Gesti�n de las dependencias}

 y gestionar algunas dependencias de librer�as, para lo cual nos serviremos de la herramienta rosdep **referencia** creada por los desarrolladores de ROS para este prop�sito.

Inicializamos rosdep y actualizamos las dependencias:

\begin{code}[htp]
\begin{lstlisting}[style=consola]
$ sudo rosdep init
$ rosdep update
\end{lstlisting}
\caption{Inicializando la herramienta \textit{rosdep}.}
\end{code}

\begin{itemize}
\item Navegaci�n: Instalamos los paquetes adicionales para realizar la navegaci�n y mapeo (se hablar� de cada uno en el cap�tulo \ref{chapter:navegacion}).
\end{itemize}

\begin{code}[!htp]
\begin{lstlisting}[style=consola]
$ sudo apt-get install ros-indigo-navigation
$ sudo apt-get install ros-indigo-gmapping
\end{lstlisting}
\caption{Instalando los paquetes de navegaci�n mapeo.}
\end{code}

\begin{itemize}
\item Rosaria: Instalamos las dependencias de RosAria, principalmente la librer�a Aria
\end{itemize}

\begin{code}[!htp]
\begin{lstlisting}[style=consola]
$ rosdep install rosaria
\end{lstlisting}
\caption{Instalando las dependencias de rosaria.}
\end{code}

\begin{itemize}
\item pocketsphinx
\item rosdeps
\item mirar cuaderno
\end{itemize}



\section{Configuraci�n del hardware}

\subsection{Calibraci�n de la odometr�a}\label{subsection:odometria}

\subsection{Ordenador de abordo}

\subsection{Sensor Kinect}

\subsection{L�ser SICK LMS100}\label{subsection:sicklms100_apendice}


\chapter{Informaci�n y documentos ONLINE}

\section{Multimedia}
\section{Repositorio de c�digo}
\subsection{Readme}
\section{Memoria del trabajo}