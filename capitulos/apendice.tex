\appendix
\addappheadtotoc
\appendixpage
\chapter{Configuraci�n del sistema} \label{chapter:configuracion_sistema}

\section{Configuraci�n del espacio de trabajo}

\subsection{Instalaci�n de las librer�as}

\subsection{Gesti�n de las dependencias}

 y gestionar algunas dependencias de librer�as, para lo cual nos serviremos de la herramienta rosdep **referencia** creada por los desarrolladores de ROS para este prop�sito.

Inicializamos rosdep y actualizamos las dependencias:

\begin{code}[htp]
\begin{lstlisting}[style=consola]
$ sudo rosdep init
$ rosdep update
\end{lstlisting}
\caption{Inicializando la herramienta \textit{rosdep}.}
\end{code}

\begin{itemize}
\item Navegaci�n: Instalamos los paquetes adicionales para realizar la navegaci�n y mapeo (se hablar� de cada uno en el cap�tulo \ref{chapter:navegacion}).
\end{itemize}

\begin{code}[!htp]
\begin{lstlisting}[style=consola]
$ sudo apt-get install ros-indigo-navigation
$ sudo apt-get install ros-indigo-gmapping
\end{lstlisting}
\caption{Instalando los paquetes de navegaci�n mapeo.}
\end{code}

\begin{itemize}
\item Rosaria: Instalamos las dependencias de RosAria, principalmente la librer�a Aria
\end{itemize}

\begin{code}[!htp]
\begin{lstlisting}[style=consola]
$ rosdep install rosaria
\end{lstlisting}
\caption{Instalando los paquetes de navegaci�n mapeo.}
\end{code}

\begin{itemize}
\item pocketsphinx
\item rosdeps
\item mirar cuaderno
\end{itemize}



\section{Configuraci�n del hardware}

\subsection{Calibraci�n de la odometr�a}\label{subsection:odometria}

\subsection{Ordenador de abordo}

\subsection{Sensor Kinect}

\subsection{L�ser SICK LMS100}\label{subsection:sicklms100}
