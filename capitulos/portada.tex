\begin{titlepage}
\begin{center}

%forma de introducir im�genes. el \\[0.5 cm] de final de l�nea introduce un salto de ese tama�o.
%width=1\textwidth indica el tama�o de la im�gen (valores entre 0-1). 
 \includegraphics[width=1\textwidth]{figuras/cabecera.png}  \\[0.5 cm]

\LARGE UNIVERSIDAD POLIT�CNICA DE MADRID \\ [1 cm]

\LARGE ESCUELA T�CNICA SUPERIOR DE INGENIER�A Y DISE�O INDUSTRIAL \\ [1 cm]

\LARGE Grado en Ingenier�a Electr�nica Industrial y Autom�tica\\ [1 cm]

\LARGE \textbf{TRABAJO FIN DE GRADO}\\[1 cm]

\Huge \textsc{Guiado de un robot m�vil basado en ROS y kinect}\\[1 cm]

\LARGE Autor: Daniel Manzaneque Amo \\[2 cm]

%flushleft alinea a la izquierda el texto

\begin{multicols}{2} 
\begin{flushleft} \Large
\emph{Cotutor:} Miguel Hernando\\
\emph{Departamento:} Electr�nica, Autom�tica e Inform�tica Industrial
\end{flushleft}

\begin{flushleft} \Large
\emph{Tutor:} Alberto Brunete\\
\emph{Departamento:} Electr�nica, Autom�tica e Inform�tica Industrial
\end{flushleft}

\end{multicols} 

%rellena de blanco el resto de la p�gina para escribir abajo del todo
\vfill

% Bottom of the page
{\large Madrid, Febrero 2016}

\end{center}
\end{titlepage}