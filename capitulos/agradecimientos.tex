\chapter{Agradecimientos}

Cuando una etapa en la vida termina normalmente uno se para y reflexiona sobre las metas y objetivos alcanzados y muchas veces descuida todo aquel esfuerzo y los momentos dif�ciles que se sortearon hasta poder cumplirlos. Quiero que estos agradecimientos sirvan para reconocer el apoyo de todas aquellas personas que han contribuido en mi vida llegado este punto y que me han ayudado de una u otra manera a superar todos los retos.\\

En primer lugar quiero agradecer el apoyo incondicional de mi familia durante todos estos a�os, en especial a mis padres por haberme dado la oportunidad de estudiar algo que me entusiasma para que en un futuro pueda dedicarme a ello. Gracias a ellos y a mi hermana por el apoyo en cada uno de los momentos complicados y por disfrutar conmigo de los logros y buenos momentos.

Quiero agradecer el apoyo de todos mis amigos pero especialmente a Samu, Manu y Sas por sus �nimos, los buenos ratos que hemos pasado y porque s� que apuestan por m� m�s de lo que hago yo a veces.

No quiero dejar pasar la oportunidad de agradecer a Andrea todo el esfuerzo y la paciencia que ha tenido conmigo desde los primeros a�os de universidad, su inestimable apoyo en los momentos de baj�n y todas las experiencias que he vivido junto a ella y que me han servido para crecer como persona, afrontar nuevos retos y creer m�s en m� cada d�a. Creo que podemos decir que los dos hemos hecho un gran doble grado despu�s de todo.

Durante mis a�os de universidad he tenido la oportunidad de compartir espacio y tiempo con grandes compa�eros. Daniela, Isma, Luis y Gema, hab�is sido grandes compa�eros y sois unos grandes amigos. Gracias por los buenos momentos en clase y sobre todo fuera de ella. Ojal� nuestras inquietudes nos lleven a trabajar juntos en futuros proyectos.

Tambi�n deseo dar las gracias a los que han sido mis compa�eros de laboratorio durante el proyecto, Mario, Irene, Sof�a, Koldo y Chechu, por todas las tardes que hemos pasado juntos y por lo mucho que nos hemos re�do. S� que me acordar� de esos momentos m�s de una vez.

Agradecer a mis profesores de instituto, en especial a mis profesores de tecnolog�a, por hacer que su labor de ese�anza me llevase a estudiar este grado. Tambi�n a mis tutores de proyecto Miguel y Alberto por haberme dado la oportunidad de desarrollar este proyecto, por haber contribuido a mi ense�anza como ingeniero y por todos sus buenos consejos.

Por �ltimo, agradecer a todos los que me han aguantado con mi robotito andando y haciendo experimentos de aqu� para all�. �Me lo he pasado en grande!
