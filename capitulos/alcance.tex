\chapter{Alcance y objetivos del proyecto}

En este cap�tulo se define el alcance y los objetivos de este proyecto, es decir, lo que se pretende conseguir con este proyecto y hasta donde puede llegar.\\

\section{Prop�sito y alcance}

El prop�sito de este proyecto es el control autom�tico de un robot m�vil utilizando ROS. Lo que se pretende es implementar la navegaci�n aut�noma del robot bas�ndose en un control reactivo a partir de la informaci�n obtenida a trav�s del sensor Kinect.\\

El alcance del proyecto requiere m�ltiples elementos de trabajo:\\

En primer lugar, requiere un conocimiento previo del sistema hardware, como es el robot Pioneer 3 AT as� como la sensor Kinect. C�mo integrar estos elementos y acceder a la informaci�n que aportan sus sensores y comandar al robot para que realice movimientos.\\

En segundo lugar, requiere un conocimiento del entorno de desarrollo ROS. Las herramientas software de las que dispone, el funcionamiento interno y la manera de programar e interaccionar con los diferentes elementos, el aprendizaje y comprensi�n.\\

En tercer lugar, incorporar los sensores pertienentes para obtener la informaci�n que permita al robot posicionarse en el entorno.\\

En cuarto lugar, implementar los ajustes necesarios para que el robot pueda operar utilizando el entorno de navegaci�n ofrecido por ROS. Realizar una configuraci�n �ptima de los sensores y realizar las pruebas reales para el c�culo de trayectorias y el control reactivo del robot.\\

Por �ltimo, realizar la integraci�n del sistema dentro de la plataforma rob�tica. Disponer de todo lo necesario para que el robot quede totalmente adaptado al sistema ROS e integrado con el sensor Kinect y los sensores pertinentes.

\section{Objetivos}

El objetivo de este proyecto es realizar el control de un robot m�vil para que sea un robot aut�nomo, bas�ndose en la informaci�n que da la odometr�a y la nube de puntos que proporciona un sensor que captura el entorno en 3 dimensiones como el sensor Kinect.
