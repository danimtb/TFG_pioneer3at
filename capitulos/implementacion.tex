\chapter{Implementaci�n del sistema}\label{chapter:implementacion}

Este cap�tulo describe los cambios, ajustes y modificaciones que, basados en la informaci�n anterior expuesta, las caracter�sticas de ROS y el hardware del que disponemos, se han realizado para alcanzar los objetivos del proyecto.

\section{Configuraciones hardware}
Como ya se ha descrito, la navegaci�n se basa en el sensor Kinect pero tambi�n se ha considerado integrar el sensor l�ser debido a la valiosa informaci�n que aporta y su disponibilidad.

Por ello, el robot deber� llevar incorporados estos sensores proporcion�ndoles alimentaci�n y una interfaz de conexi�n adecuada.

\subsection{Pioneer 3 AT}
El robot Pioneer 3 AT de Adept Mobile Robots es la base de la plataforma rob�tica. El modelo disponible en el laborarotio de la Escuela T�cnica Superior de Ingenier�a y Dise�o Industrial llevaba incorporado un ordenador de tipo **ORDENADOR PIONEER**. Tambi�n, al comienzo de este proyecto ya exist�an algunas adaptaciones como la incorporaci�n de un altavoz frontal, acceso a los puestos USB del ordenador interno y conexi�n para el sensor l�ser.

IMAGEN ROBOT ESTADO INICIAL.

El robot hab�a sido utilizado anteriormente mediante el software MRCore en el proyecto **PROYECTO DE ALEJANDRO** y el sistema operativo del ordenador interno era Ubuntu Server 10.

Para integrar ROS
\subsection{Sensor Kinect}
\subsection{Sensor L�ser}
\subsection{Intel NUC}

\subsection{Primera configuraci�n}
\subsection{Segunda configuraci�n}

\section{Navegaci�n}
\subsection{Sensores}
\subsection{Navegaci�n con mapa}
\subsection{Navegaci�n reactiva}
\section{Nodo de navegaci�n por puntos}
\section{Nodo de comandos por voz}
\subsection{Reconocimiento de comandos de voz}
\subsection{Feedback mediante text-to-speech}

\section{Nodo de ejecuci�n autom�tica de nodos}
