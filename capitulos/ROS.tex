\chapter{Entorno ROS}

La versi�n del software ROS utilizada para el desarrollo del proyecto ha tratado de ser siempre la m�s actual posible, ya que eso nos asegura mantener la compatibilidad en futuros trabajos y que el software est� actualizado.

La versi�n utilizada fue ROS Indigo Igloo desde el comienzo del proyecto, bajo el sistema operativo Ubuntu 14.04. A fecha de entrega del proyecto existe una versi�n m�s actualizada del software ROS, sin embargo se desestim� su uso debido a que a�n no era una versi�n estable y algunos paquetes no se encontraban disponibles para tal versi�n.

ROS realiza funciones similares a un sistema operativo, como la comunicaci�n e interacci�n entre diferentes procesos, la dis tribuci�n en hilos, comunicaci�n distribuida entre m�quinas...

El concepto m�s importante dentro de ROS son los nodos, que no son m�s que un proceso que se ejecuta y conecta al proceso principal, llamado m�ster, para comunicarse con otros nodos. Existen diferentes conceptos rostopics, rosservices, nodelets... que ser�n explicados en el siguiente punto.

\subsection{Funcionamiento de ROS}

- Nodos.
- Servicios.
- Rostopics.

\section{Configuraci�n de ROS}
Para la instalaci�n de ROS es necesario seguir ciertos pasos bien explicados en la wiki de su p�gina.

Para el ordenador de abordo del robot utilizamos la versi�n completa del software.

A�adimos nuestra variable de entorno para buscar en los directorios de ros cuando ejecutemos los nodos.

Para el desarrollo dentro de ROS se utiliza el entorno de desarrollo Catkin, que facilita el enlazado y compilaci�n de los paquetes. para ello es necesario tenerlo instalado:

**instalacion catkin**

Y a continuaci�n es necesario incluir nuestro directorio de desarrollo para que sea reconocido:

**source**

A partir de este punto podr�amos realizar nodos utilizando las funcionalidades de ROS.

Para el desarrollo de este proyecto es necesario instalar las siguientes funcionalidades adicionales de ROS:

- Navegacion
- Rosaria
- pocketsphinx
- rosdeps
- mirar cuaderno

\section{Configuraci�n de los paquetes ROS}

Los paquetes ROS son funcionalidades desarrolladas por terceros que se integran en el sistema ROS y que son transferibles de un robot a otro.

Los paquetes ROS incorporan un archivo CMakeLists.txt para la compilaci�n de los nodos que se hayan desarrollado as� como sus mensajes. Y un archivo Package Manifest package.xml, donde se indican los requisitos del paquete, el autor, el contacto, y las dependencias del mismo.
