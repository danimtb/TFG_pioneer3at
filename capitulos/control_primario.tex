\chapter{Control primario}

En este cap�tulo se realiza una primera aproximaci�n al control del robot y a los nodos b�sicos que deben ejecutarse para controlarlo y acceder a la informaci�n de los sensores.

\section{Nodos hardware}

Los nodos necesarios para el control del robot requieren el acceso a los motores y a la lectura de la odometr�a del robot. Tambi�n es necesario disponer de controladores para ambos sensores utilizados y qe su informaci�n se publique en tipos de datos reconocibles por ROS. Los nodos utilizados se describen a continuaci�n. 

\subsection{Control del robot: Rosaria y p2os}

Como hemos indicado anteriormente, la librer�a que nos proporciona el acceso a la placa controladora de nuestro robot Pioneer es Aria. Esta librer�a es la que proporciona Adept Mobile Robots para realizar el control completo del robot y acceder a sus par�metros configurables.

Los paquetes disponibles en ROS para el control de los robots de la familia Pioneer son dos, por un lado tenemos Rosaria y por otro p2os.

p2os es un paquete que agrupa conjunto de utilidades y nodos desarrollados para controlar el robot. Su caracter�stica principal es que accede de manera nativa a la placa controladora del robot y no dependen de la librer�a Aria. Adem�s incorpora funcionalidades configuradas como modelos 3D de robot, simulaci�n con Gazebo o la configuraci�n de la navegaci�n.

Sin embargo, p2os no integra todas las funcionalidades a las que tiene acceso Aria como son la reconfiguraci�n de los par�metros de la odometr�a.

Rosaria es un nodo de interfaz entre ROS y Aria, por tanto incluye todas pr�cticamente todas las funcionalidades de esta. Podemos acceder a la calibraci�n de los encoders de la odometr�a as� como conectar con el simulador MobileSim (ver secci�n \ref{MobileSim})

A continuaci�n se muestra el launchfile para ejecutar el nodo RosAria:

\begin{code}[htp]
\begin{lstlisting}[style=launch]
<launch>
<!-- Starting rosaria driver for motors and encoders -->
  <node name="rosaria" pkg="rosaria" type="RosAria" args="_port:=/dev/ttyUSB0">
  <rosparam>
      TicksMM: 166
      RevCount: 37350
      DriftFactor: 0
  </rosparam>
  <remap from="~cmd_vel" to="cmd_vel"/>
  </node>
</launch>
\end{lstlisting}
\hypersetup{urlcolor=black}
Fuente: \href{https://github.com/danimtb/pioneer3at_ETSIDI/blob/master/pioneer_utils/sensors/rosaria.launch}{\textit{ pioneer\_utils/sensors/rosaria.launch}}
\hypersetup{urlcolor=blue}
\caption{Launchfile para RosAria.}
\end{code}

Como puede verse, podemos modificar los valores usados por Aria para realizar el c�mputo de la odometr�a.

\subsection{Sensor Kinect}


\subsection{Sensor L�ser}

\section{Nodo de teleoperaci�n}

\section{Nodo de navegaci�n estimada}

\section{Nodo de guiado (follower)}
